\documentclass[12pt]{article}

\usepackage{amsmath}
\usepackage{graphicx}
\usepackage{subfigure}
\usepackage{float}
\usepackage{ulem}
\usepackage{bm}
\usepackage{anysize}

\marginsize{2cm}{2cm}{0.9cm}{1.8cm}

\title{EECE 5639 Computer Vision\\ [2ex] \begin{large} Homework \#1 \end{large} }
\author{Jiyu Tian}
\date{}

\begin{document}
\maketitle
%%---------------------------------------------------------------
%% Question 1
%%---------------------------------------------------------------
\section{Solution:}
(a) The equations of the image of the given line is:
\begin{equation*}
x = f\frac{X_0 + \lambda(X_1 - X_0)}{Z_0 + \lambda(Z_1 - Z_0)}
\end{equation*}
\begin{equation*}
y = f\frac{Y_0 + \lambda(Y_1 - Y_0)}{Z_0 + \lambda(Z_1 - Z_0)}
\end{equation*}
\begin{equation*}
z = f
\end{equation*}
\noindent(b) As $|\lambda| \to \infty$:
\begin{equation*}
\lim\limits_{\lambda\to\infty}x = f\frac{X_1 - X_0}{Z_1 - Z_0}
\end{equation*}
\begin{equation*}
\lim\limits_{\lambda\to\infty}y = f\frac{Y_1 - Y_0}{Z_1 - Z_0}
\end{equation*}
\begin{equation*}
\lim\limits_{\lambda\to\infty}z = f
\end{equation*}
\noindent If the 3D line is not parallel to the image plane, $Z_0 \neq Z_1$. The image of $P$ will converge to a point which depends on the relative position of $P_0$ and $P_1$.\\\\
(c) Let $\lambda = 1/2$, the image of the midpoint of the segment $P_0P_1$ is 
\begin{equation*}
x_{midP_0P_1} = f\frac{X_0 + X_1}{Z_0 + Z_1}
\end{equation*}
\begin{equation*}
y_{midP_0P_1} = f\frac{Y_0 + Y_1}{Z_0 + Z_1}
\end{equation*}
\begin{equation*}
z_{midP_0P_1} = f
\end{equation*}
\noindent However the midpoint of the image segment $p_0p_1$ is
\begin{equation*}
x_{midp_0p_1} = \frac{f}{2}\left ( \frac{X_0}{Z_0} + \frac{X_1}{Z_1} \right )
\end{equation*}
\begin{equation*}
y_{midp_0p_1} = \frac{f}{2}\left ( \frac{Y_0}{Z_0} + \frac{Y_1}{Z_1} \right )
\end{equation*}
\begin{equation*}
z_{midp_0p_1} = f
\end{equation*}
\noindent If the 3D line is not parallel to the image plane, $Z_0 \neq Z_1$. Then $x_{midp_0p_1} \neq x_{midP_0P_1},\ y_{midp_0p_1} \neq y_{midP_0P_1}$,  the image of the midpoint of the segment $P_0P_1$ is not the midpoint of the image segment $p_0p_1$.

\vfill
\clearpage
%%---------------------------------------------------------------
%% Question 2
%%---------------------------------------------------------------
\section{Solution:}
From the picture we have
\begin{equation*}
\begin{aligned}
x_c &= - y_w + 2\\
y_c &= z_w - 2\\
z_c &= - x_w + 10
\end{aligned}
\end{equation*}
And thus
\begin{equation*}
M_{ext} = \left[ \begin{array}{cccc}
0 & -1 & 0 & 2\\
0 & 0 & 1 & -2\\
-1 & 0 & 0 & 10\\
0 & 0 & 0 & 1
\end{array} \right ]
\end{equation*}
Line equation for those sides of the floor tiles parallel to the camera optical axis can be given in world coordinate:
\begin{equation*}
P_w = \left[ \begin{array}{c}
-\lambda\\
n\\
0\\
1
\end{array} \right ]
\end{equation*}
where $n \in \{ 0,\ 0.5,\ 1,\ 1.5,\ 2,\ 2.5,\ 3,\ 3.5,\ 4 \}$ denotes each line of tiles, and $\lambda \in R$ denotes the points on line. World coordinate to camera coordinate is given by
\begin{equation*}
P_c = M_{ext} \cdot P_w = \left[ \begin{array}{cccc}
0 & -1 & 0 & 2\\
0 & 0 & 1 & -2\\
-1 & 0 & 0 & 10\\
0 & 0 & 0 & 1
\end{array} \right ]\left[ \begin{array}{c}
-\lambda\\
0.5n\\
0\\
1
\end{array} \right ] = \left[ \begin{array}{c}
-n+2\\
-2\\
\lambda + 10\\
1
\end{array} \right ]
\end{equation*}
And accordingly on the image plane,
\begin{equation*}
\left[ \begin{array}{c}
x'\\
y'\\
z'
\end{array} \right ] = \left[ \begin{array}{cccc}
f & 0 & 0 & 0\\
0 & f & 0 & 0\\
0 & 0 & 1 & 0
\end{array} \right ]  \left[ \begin{array}{c}
-n+2\\
-2\\
\lambda + 10\\
1
\end{array} \right ] = \left[ \begin{array}{c}
(-n+2)f\\
-2f\\
\lambda + 10
\end{array} \right ]
\end{equation*}
\begin{equation*}
P_i = \left[ \begin{array}{c}
x\\
y
\end{array} \right ] = \left[ \begin{array}{c}
x'/z'\\
y'/z'
\end{array} \right ] = \left[ \begin{array}{c}
\frac{-n+2}{\lambda + 10}f\\
-\frac{2}{\lambda + 10}f
\end{array} \right ]
\end{equation*}
As $|\lambda| \to \infty$,
\begin{equation*}
\lim\limits_{\lambda\to\infty}P_i = \lim\limits_{\lambda\to\infty}\left[ \begin{array}{c}
\frac{-n+2}{\lambda + 10}f\\
-\frac{2}{\lambda + 10}f
\end{array} \right ] = \left[ \begin{array}{c}
0\\
0
\end{array} \right ]
\end{equation*}
Therefore, the images of the sides of the floor tiles parallel to the camera optical axis intersect at a single point $[0,\ 0]'$, i.e. the origin.
\vfill
\clearpage
%%---------------------------------------------------------------
%% Question 3
%%---------------------------------------------------------------
\section{Solution:}
Suppose a point on the line is $P_0 = (x_0,\ y_0,\ z_0)$, and a vector parallel to the line is $v = (v_x\  v_y,\ v_z)$. Then the line equation in 3D under camera coordinate is 
\begin{equation*}
P = P_0 + \lambda v = \left[ \begin{array}{c}
x_0 + \lambda v_x\\
y_0 + \lambda v_y\\
z_0 + \lambda v_z
\end{array} \right ]
\end{equation*}
The image of the 3D line is 
\begin{equation*}
\left[ \begin{array}{c}
x'\\
y'\\
z'
\end{array} \right ] = \left[ \begin{array}{cccc}
f & 0 & 0 & 0\\
0 & f & 0 & 0\\
0 & 0 & 1 & 0
\end{array} \right ] \left[ \begin{array}{c}
x_0 + \lambda v_x\\
y_0 + \lambda v_y\\
z_0 + \lambda v_z\\
1
\end{array} \right ] = \left[ \begin{array}{c}
(x_0 + \lambda v_x)f\\
(y_0 + \lambda v_y)f\\
z_0 + \lambda v_z
\end{array} \right ]
\end{equation*}
\begin{equation*}
x = \frac{x'}{z'} = \frac{x_0 + \lambda v_x}{z_0 + \lambda v_z}f, \ y = \frac{y'}{z'} = \frac{y_0 + \lambda v_y}{z_0 + \lambda v_z}f
\end{equation*}
As $|\lambda| \to \infty$,

\begin{equation*}
\begin{aligned}
\lim\limits_{\lambda\to\infty}x &= \lim\limits_{\lambda\to\infty}\frac{x_0 + \lambda v_x}{z_0 + \lambda v_z}f = \frac{v_x}{v_z}f\\
\lim\limits_{\lambda\to\infty}y &= \lim\limits_{\lambda\to\infty}\frac{y_0 + \lambda v_y}{z_0 + \lambda v_z}f = \frac{v_y}{v_z}f
\end{aligned}
\end{equation*}
Since the vanishing point under camera coordinate is $(10, 0, f)$, we have
\begin{equation*}
\left[ \begin{array}{c}
\frac{v_x}{v_z}f=10\\
\frac{v_y}{v_z}f=0
\end{array} \right ]
\end{equation*}
Let $v_z = 1$, we have the orientation vector
\begin{equation*}
v = \left[ \begin{array}{c}
10/f\\
0\\
1
\end{array} \right ]
\end{equation*}
with respect to the camera coordinate system.
%%---------------------------------------------------------------
%% Question 4
%%---------------------------------------------------------------
\section{Solution:}
\noindent Suppose $Z_{face} - Z_{nose} = \Delta Z$. When taken frontally, from the perspective equation:
\begin{equation*}
x_{nose} = f\frac{X_{nose}}{Z_{nose}},\ x_{face} = f\frac{X_{face}}{Z_{face}}
\end{equation*}
\noindent The ratio nose appears to be is 
\begin{equation*}
\frac{x_{nose}}{x_{face}} = \frac{X_{nose}Z_{face}}{Z_{nose}X_{face}} = \frac{X_{nose}}{X_{face}} \left(1 + \frac{\Delta Z}{Z_{nose}}\right )
\end{equation*}
\noindent where $X_{face},\ X_{nose}$ and $\Delta Z$ keep the same. When taken from a small distance, the nose is closer to the camera and $\Delta Z / Z_{nose}$ can be rather large. Therefore the nose appears to be much larger than the rest of the face.\\\\
This cannot be reduced by using another focal length, because $f$ has no effect in the ratio. However this can be reduced by changing object distance, i.e. to take picture in a larger distance.
\vfill
\clearpage
%%---------------------------------------------------------------
%% Question 5
%%---------------------------------------------------------------
\section{Solution:}
Red and magenta appear almost the same to person with a blue receptor deficiency.
%%---------------------------------------------------------------
%% Question 6
%%---------------------------------------------------------------
\section{Solution:}
(a) CMY system\\
\begin{equation*}
{\left[ \begin{array}{c}
C\\
M\\
Y
\end{array} 
\right]} = {\left[ \begin{array}{c}
1\\
1\\
1
\end{array} 
\right]} - 
{\left[ \begin{array}{c}
R\\
G\\
B
\end{array} 
\right]} = 
{\left[ \begin{array}{c}
1\\
1\\
1
\end{array} 
\right]} - \frac{1}{255}
{\left[ \begin{array}{c}
200\\
50\\
100
\end{array} 
\right]} 
={\left[ \begin{array}{c}
0.216\\
0.804\\
0.608
\end{array} 
\right]}
\end{equation*}

\noindent(b) YIQ system
\begin{equation*}
{\left[ \begin{array}{c}
Y\\
I\\
Q
\end{array} 
\right]} = {\left[ \begin{array}{ccc}
0.299 & 0.587 & 0.114\\
0.596 & -0.275 & -0.321\\
0.212 & -0.532 & 0.311
\end{array} 
\right]}  
{\left[ \begin{array}{c}
R\\
G\\
B
\end{array} 
\right]} = 
\frac{1}{255}
{\left[ \begin{array}{ccc}
0.299 & 0.587 & 0.114\\
0.596 & -0.275 & -0.321\\
0.212 & -0.532 & 0.311
\end{array} 
\right]} {\left[ \begin{array}{c}
200\\
50\\
100
\end{array} 
\right]} 
={\left[ \begin{array}{c}
0.394\\ 
0.288\\
0.184
\end{array} 
\right]}
\end{equation*}

\noindent(c) HSI system
\begin{equation*}
\theta = cos^{-1}\left(\frac{[(R-G)+(R-B)]/2}{[(R-G)^2+(R-B)(G-B)]^{1/2}}\right)=cos^{-1}\frac{125}{\sqrt{17500}}=19.107
\end{equation*}
\begin{equation*}
H = 360 - \theta = 340.893
\end{equation*}
\begin{equation*}
S = 1-\frac{3}{R+G+B}[min(R, G, B)] = 1- \frac{3\times50}{350} = 0.571
\end{equation*}

\begin{equation*}
I = 1-\frac{R+G+B}{3} = 1-\frac{350}{3\times255} = 0.673
\end{equation*}
\begin{equation*}
{\left[ \begin{array}{c}
H\\ 
S\\
I
\end{array} 
\right]} = {\left[ \begin{array}{c}
340.893\\ 
0.571\\
0.673
\end{array} 
\right]}
\end{equation*}

\end{document}
