\documentclass[10pt,landscape]{article}
\usepackage{multicol}
\usepackage{calc}
\usepackage{ifthen}
\usepackage[landscape]{geometry}
\usepackage{hyperref}
\usepackage{amsmath}
\usepackage{graphicx}
\usepackage{subfigure}
\usepackage{float}
\usepackage{ulem}
\usepackage{bm}
\usepackage{anysize}
\usepackage{mathrsfs}
% To make this come out properly in landscape mode, do one of the following
% 1.
%  pdflatex latexsheet.tex
%
% 2.
%  latex latexsheet.tex
%  dvips -P pdf  -t landscape latexsheet.dvi
%  ps2pdf latexsheet.ps


% If you're reading this, be prepared for confusion.  Making this was
% a learning experience for me, and it shows.  Much of the placement
% was hacked in; if you make it better, let me know...


% 2008-04
% Changed page margin code to use the geometry package. Also added code for
% conditional page margins, depending on paper size. Thanks to Uwe Ziegenhagen
% for the suggestions.

% 2006-08
% Made changes based on suggestions from Gene Cooperman. <gene at ccs.neu.edu>


% To Do:
% \listoffigures \listoftables
% \setcounter{secnumdepth}{0}


% This sets page margins to .5 inch if using letter paper, and to 1cm
% if using A4 paper. (This probably isn't strictly necessary.)
% If using another size paper, use default 1cm margins.
\ifthenelse{\lengthtest { \paperwidth = 11in}}
	{ \geometry{top=.5in,left=.5in,right=.5in,bottom=.5in} }
	{\ifthenelse{ \lengthtest{ \paperwidth = 297mm}}
		{\geometry{top=1cm,left=1cm,right=1cm,bottom=1cm} }
		{\geometry{top=1cm,left=1cm,right=1cm,bottom=1cm} }
	}

% Turn off header and footer
\pagestyle{empty}
 

% Redefine section commands to use less space
\makeatletter
\renewcommand{\section}{\@startsection{section}{1}{0mm}%
                                {-1ex plus -.5ex minus -.2ex}%
                                {0.5ex plus .2ex}%x
                                {\normalfont\large\bfseries}}
\renewcommand{\subsection}{\@startsection{subsection}{2}{0mm}%
                                {-1explus -.5ex minus -.2ex}%
                                {0.5ex plus .2ex}%
                                {\normalfont\normalsize\bfseries}}
\renewcommand{\subsubsection}{\@startsection{subsubsection}{3}{0mm}%
                                {-1ex plus -.5ex minus -.2ex}%
                                {1ex plus .2ex}%
                                {\normalfont\small\bfseries}}
\makeatother

% Define BibTeX command
\def\BibTeX{{\rm B\kern-.05em{\sc i\kern-.025em b}\kern-.08em
    T\kern-.1667em\lower.7ex\hbox{E}\kern-.125emX}}

% Don't print section numbers
\setcounter{secnumdepth}{0}


\setlength{\parindent}{0pt}
\setlength{\parskip}{0pt plus 0.5ex}


% -----------------------------------------------------------------------

\begin{document}

\raggedright
\footnotesize
\begin{multicols}{3}


% multicol parameters
% These lengths are set only within the two main columns
%\setlength{\columnseprule}{0.25pt}
\setlength{\premulticols}{1pt}
\setlength{\postmulticols}{1pt}
\setlength{\multicolsep}{1pt}
\setlength{\columnsep}{2pt}

\begin{center}
     \Large{\textbf{Computer Vision Midterm2}} \\
\end{center}

%------------------------------
\section{Stereo Vision}
The ability to infer information on the 3D structure and distance of a scene from two or more images taken from different viewpoints.
%------------------------------
%------------------------------
\subsection{A simple stereo system}
\begin{equation*}
    \frac{T+x_l-x_r}{Z-f} = \frac{T}{Z},\ Z = f\frac{T}{x_r-x_l}, \ Z = f\frac{T}{d}
\end{equation*}
\textbf{Disparity}: $d=x_r-x_l$



%------------------------------
%------------------------------
\subsection{Epipolar Geometry}
Epipoles:
\begin{itemize}
    \item  $e_l$: left image of $O_r$
    \item  $e_r$: right image of $O_l$
\end{itemize}

Epipolar plane:
\begin{itemize}
    \item Three points: $O_l$, $O_r$, and $P$ define an epipolar plane
\end{itemize}

Epipolar lines and epipolar constraint:
\begin{itemize}
    \item Intersections of epipolar plane with the image planes
    \item Corresponding points are on ``conjugate'' epipolar lines 
\end{itemize}


Find Epipoles, given $p_l$:
\begin{itemize}
    \item consider its epipolar line: $p_l, e_l$
    \item find epipolar plane: $O_l, p_l, e_l$
    \item intersect the epipolar plane with the right image plane
    \item search for $p_r$ on the right epipolar line
\end{itemize}
Parallel Cameras: 
\begin{itemize}
    \item Epipoles are at infinity
    \item Epipolar lines are parallel to the baseline
\end{itemize}


%------------------------------
%------------------------------
\subsection{Essential Matrix}

The \textbf{ESSENTIAL} matrix is a $3\times3$ matrix that “encodes” the epipolar geometry of two views. Given a point in an image, multiplying by the Essential Matrix, will tell us the \textbf{EPIPOLAR} line in the second image where the corresponding point must be.\\
\begin{equation*}
    E=RS,\ P^T_rEP_l = 0,\ p^T_rEp_l = 0 
\end{equation*}
Essential matrix has rank 2, and depends only on the EXTRINSIC Parameters ($R$ \& $T$).
\begin{equation*}
T =\left[ \begin{array}{ccc}
T_x & T_y & T_z
\end{array} \right ], S = \left[ \begin{array}{ccc}
0 & -T_z & T_y\\
T_z & 0 & -T_x\\
-T_y & T_x & 0
\end{array} \right ]
\end{equation*}
$p_r$ belongs to epipolar line in the right image defined by
\begin{equation*}
    l_r = Ep_l
\end{equation*}
$p_l$ belongs to epipolar line in the left image defined by 
\begin{equation*}
    l_l = E^Tp_r
\end{equation*}
Epipoles belong to the epipolar lines, and they belong to all the epipolar lines:
\begin{equation*}
    e^T_rE=0, Ee_l=0
\end{equation*}



%------------------------------
%------------------------------
\subsection{Fundamental Matrix}
The essential matrix uses \textbf{CAMERA} coordinates. To use image coordinates we must consider the \textbf{INTRINSIC} camera parameters.
\begin{equation*}
    p_l=M_l^{-1}\bar{p_l},\ p_r=M_r^{-1}\bar{p_r}, \bar{p_r}^TF\bar{p_l}=0,\ F=M_r^{-T}RSM^{-1}_l
\end{equation*}


Fundamental matrix has rank 2, and depends on the INTRINSIC and EXTRINSIC Parameters ($f$, etc ; $R$ \& $T$). Analogous to the Essential matrix, the Fundamental matrix also tells how points in each image are related to epipolar lines in the other image.


%------------------------------
%------------------------------
\subsection{Eight Points Algorithm}
\begin{itemize}
    \item $F$ is a $3\times3$ matrix (9 entries) but rank 2
    \item \textbf{HOMOGENEOUS} linear system with 9 unknowns
    \item Need $m \geq 8$; solution will be up to a constant
\end{itemize}
\begin{equation*}
    \left[ \begin{array}{ccc}
x_1x'_1 & ... & x_mx'_m\\
 x_1y'_1 & ... & x_my'_m \\
 x_1 & ... & x_m\\
 y_1x'_1 & ... & y_mx'_m\\
 y_1y'_1 & ... & y_my'_m\\
 y_1 & ... & y_m\\
 x'_1 & ... & x'_m\\
 y'_1 & ... & y'_m\\
 1 & ... & 1
    \end{array} \right ]^T    \left[ \begin{array}{c}
f_11\\
f_21\\
f_31\\
f_12\\
f_22\\
f_32\\
f_13\\
f_23\\
f_33
    \end{array} \right ] = 0
\end{equation*}
Assume that we need to find the non trivial solution of:
\begin{equation*}
    Ax=0
\end{equation*}
with $m$ equations and $n$ unknowns, $m \geq n – 1$ and rank$(A) = n-1$. Since the norm of $x$ is arbitrary, we will look for a solution
with norm $||x|| = 1$. We want $Ax$ as close to $0$ as possible and $||x|| =1$:
\begin{equation*}
    \min_x ||Ax||^2\ \ \text{s.t. } ||x||^2 = 1,\ ||Ax||^2 = x^TA^TAx
\end{equation*}
Define the following cost:
\begin{equation*}
\mathscr{L}(x) = x^TA^TAx - \lambda(x^Tx - 1)
\end{equation*}
This cost is called the\textbf{ LAGRANGIAN cost} and $\lambda$ is called the \textbf{LAGRANGIAN multiplier}. The Lagrangian incorporates the constraints into the cost function by introducing extra variables.
\begin{itemize}
    \item Construct the $m \times 9$ matrix $A$
    \item Find the SVD of $A$: $A = UDV^T$
    \item The columns of $V$ are the eigenvectors of $A^TA$; the last one corresponds to the smallest eigenvalue
    \item The entries of $F$ are the components of the last column of $V$ corresponding to the least s.v.
\end{itemize}
F must be singular. To enforce it:
\begin{itemize}
    \item Find the SVD of $F$: $F = U_fD_fV_f^T$
    \item Set smallest s.v. of $F$ to $0$ to create $D’_f$
    \item Recompute $F: F = U_fD’_fV_f^T$
\end{itemize}



%------------------------------
%------------------------------
\subsection{3D Reconstruction}
Intrinsic:
\begin{itemize}
    \item $f_1$ and $f_2$: focal lengths
    \item $c_1$ and $c_2$: principal points
    \item Pixel size
\end{itemize}
•Extrinsic
\begin{itemize}
    \item Transformation $(R,T)$ between cameras
\end{itemize}


%------------------------------
\section{Motion}

%------------------------------
%------------------------------
\subsection{Time to Collision}
An object of height $L$ moves with constant velocity $v$. 
\begin{itemize}
    \item At time $t=0$ the object is at $D(0) = D_0$
    \item At time $t$ it is at $D(t) = D_0 - vt$
    \item It will crash with the camera at time $\tau = D_0/v$
\end{itemize}


%------------------------------
%------------------------------
\subsection{Comparison}
Stereo:
\begin{itemize}
    \item Two or more frames
    \item Baseline is usually larger
    \item Stereo images are taken at the same time
\end{itemize}

Motion
\begin{itemize}
    \item $N$ frames
    \item Motion disparities tend to be smaller
    \item Motion disparities can be due to scene motion
    \item There can be more than 1 transf between frames
\end{itemize}




%------------------------------
%------------------------------
\subsection{Motion Field (MF)}
The MF assigns a velocity vector to each pixel in the image.\\
These velocities are INDUCED by the RELATIVE MOTION btw the camera and the 3D scene.\\
The MF can be thought as the \textit{projection} of the 3D velocities on the image plane.\\
The relative velocity of $P$ $w.r.t.$ camera:
\begin{equation*}
    V = - T - \omega \times P,\ T = \left[ \begin{array}{c}
    T_x\\
    T_y\\
    T_z
\end{array} \right ],\ \omega = \left[ \begin{array}{c}
    \omega_x\\
    \omega_y\\
    \omega_z
\end{array} \right ]
\end{equation*}
Translation velocity and rotation angular velocity.\\


\subsubsection{The velocity of p}
\begin{equation*}
    v = f\frac{V}{Z} - p\frac{V_z}{Z}
\end{equation*}

\begin{equation*}
    v_x = \frac{T_zx - T_xf}{Z} - \omega_yf + \omega_zy + \frac{\omega_xxy}{f} - \frac{\omega_yx^2}{f}
\end{equation*}

\begin{equation*}
    v_y = \frac{T_zy - T_yf}{Z} + \omega_xf - \omega_zx - \frac{\omega_yxy}{f} + \frac{\omega_xy^2}{f}
\end{equation*}

Translational component and rotational component. The rotational component is independent of depth $Z$!

\subsubsection{Pure Translation}


\begin{equation*}
    v_x = (x-x_0)\frac{T_z}{Z} ,\ v_y = (y-y_0)\frac{T_z}{Z} 
\end{equation*}
The motion field in this case is RADIAL. It consists of vectors passing through $p_0 = (x_0,y_0)$.

$Tz > 0$, (camera moving towards object)
\begin{itemize}
    \item the vectors point away from $p_0$
    \item $p_0$ is the POINT OF EXPANSION
\end{itemize}

$Tz < 0$, (camera moving away from object)
\begin{itemize}
    \item the vectors point towards $p_0$
    \item $p_0$ is the POINT OF CONTRACTION
\end{itemize}

If $T_z\neq 0$ the MF is RADIAL with all vectors pointing towards (or away from) a single point $p_0$.
If $T_z = 0$, all motion field vectors are parallel to each other and inversely proportional to depth. 
The length of the MF vectors is inversely proportional to depth $Z$.
If $T_z \neq 0$ it is also directly proportional to the distance between $p$ and $p_0$.
\begin{equation*}
    p_0 = \left[ \begin{array}{c}
         x_0 \\
         y_0\\
         f
    \end{array}\right]= \left[ \begin{array}{c}
         fT_x/T_z \\
         fT_y/T_z\\
         f
    \end{array}\right]
\end{equation*}
$p_0$ is the vanishing point of the direction of translation.
$p_0$ is the intersection of the ray parallel to the translation vector and the image plane.


%------------------------------
%------------------------------
\subsection{OPTICAL FLOW}
We will use the apparent motion of brightness patterns observed in an image sequence. This motion is called OPTICAL FLOW.
\textbf{MF $\neq$ OF.} Consider a smooth, lambertian, uniform sphere rotating around a diameter, in front of a camera. MF $\neq 0$ since the points on the sphere are moving; OF $= 0$ since there are no moving patterns in the images. \\
\textbf{MF $\neq$ OF.} Consider a still, smooth, specular, uniform sphere, in front of a stationary camera and a moving light source. MF $= 0$ since the points on the sphere are not moving; OF $\neq 0$  since there is a moving pattern in the images. \\
Never the less, keeping in mind that MF $\neq$ OF, we will assume that the apparent brightness of moving objects remains constant and hence we will estimate OF instead (since MF cannot really be observed!).\\
The Image Brightness Constancy Assumption only provides the OF component in the direction of the spatial image gradient.
\subsubsection{Differential Techniques}
Based on spatial and temporal variations of the image brightness at all pixels. Used to compute DENSE OF.\\
Assumptions:
\begin{itemize}
    \item Brightness Constancy: $(\nabla E)^Tv+E_t = \epsilon \sim 0$, where ε accounts for noise and model errors
    \item OF is constant in small patches
\end{itemize}
\begin{equation*}
\left[ \begin{array}{cc}
\sum I^2_x & \sum I_xI_y\\
\sum I_xI_y & \sum I^2_y
\end{array} \right] \left[ \begin{array}{c}
V_x\\
V_y
\end{array} \right]= -\left[ \begin{array}{c}
\sum I_xI_t\\
\sum I_yI_t
\end{array} \right]
\end{equation*}
Aperture Problem:
\begin{itemize}
    \item One $e.v. = 0$ -> no corner, just an edge. At edges, $A^TA$ becomes singular.
    \item Two $e.v. = 0$ -> no corner, homogeneous region. At homogeneous regions, $A^TA$ becomes 0 (0 eigenvalues).
\end{itemize}

\subsubsection{Matching Techniques}
Estimates OF only at localized features. Produces SPARSE OF mappings.\\
Fails for large motions. Traditional LK is just refining a position estimate. We must be close to the right answer, and it only finds a local min. We can increase range of LK methods by using a coarse to fine image pyramid, so at each level the refinement is small. \\
If object moves fast, the brightness changes quickly and small masks fail to estimate the spatiotemporal derivatives. Pyramids can be used to compute large optical flow vectors.\\
\textbf{Pyramids} is very useful for image representation. Built by using multiple copies at different resolutions. Each level in the pyramid is $1/4$ size of the previous level (half columns, half rows). The lowest level has the highest resolution. The highest level has the lowest resolution.\\
Blur the image using a Gaussian filter, before \textbf{down-sampling}. Throw away every other row and column to create an image at $1/2$ of the scale. It happens when your sampling rate is not high enough to capture the amount of detail change.\\
\textbf{Up-sampling}. Interpolation via Convolution: Nearest-neighbor / Linear / Gaussian interpolation.
%------------------------------
\section{Tracking}
Basic Template Matching Assumptions:
\begin{itemize}
    \item A snapshot of the target can be used to describe it
    \item Target does not change (much) between frames
    \item Motion is mostly translational
\end{itemize}





\end{multicols}
\end{document}
