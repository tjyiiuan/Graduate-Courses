\documentclass[10pt,landscape]{article}
\usepackage{multicol}
\usepackage{calc}
\usepackage{ifthen}
\usepackage[landscape]{geometry}
\usepackage{hyperref}
\usepackage{amsmath}
\usepackage{graphicx}
\usepackage{subfigure}
\usepackage{float}
\usepackage{ulem}
\usepackage{bm}
\usepackage{anysize}
\usepackage{mathrsfs}


\ifthenelse{\lengthtest { \paperwidth = 11in}}
	{ \geometry{top=.5in,left=.5in,right=.5in,bottom=.5in} }
	{\ifthenelse{ \lengthtest{ \paperwidth = 297mm}}
		{\geometry{top=1cm,left=1cm,right=1cm,bottom=1cm} }
		{\geometry{top=1cm,left=1cm,right=1cm,bottom=1cm} }
	}

% Turn off header and footer
\pagestyle{empty}
 

% Redefine section commands to use less space
\makeatletter
\renewcommand{\section}{\@startsection{section}{1}{0mm}%
                                {-1ex plus -.5ex minus -.2ex}%
                                {0.5ex plus .2ex}%x
                                {\normalfont\large\bfseries}}
\renewcommand{\subsection}{\@startsection{subsection}{2}{0mm}%
                                {-1explus -.5ex minus -.2ex}%
                                {0.5ex plus .2ex}%
                                {\normalfont\normalsize\bfseries}}
\renewcommand{\subsubsection}{\@startsection{subsubsection}{3}{0mm}%
                                {-1ex plus -.5ex minus -.2ex}%
                                {1ex plus .2ex}%
                                {\normalfont\small\bfseries}}
\makeatother

% Define BibTeX command
\def\BibTeX{{\rm B\kern-.05em{\sc i\kern-.025em b}\kern-.08em
    T\kern-.1667em\lower.7ex\hbox{E}\kern-.125emX}}

% Don't print section numbers
\setcounter{secnumdepth}{0}


\setlength{\parindent}{0pt}
\setlength{\parskip}{0pt plus 0.5ex}


% -----------------------------------------------------------------------

\begin{document}

\raggedright
\footnotesize
\begin{multicols}{3}


% multicol parameters
% These lengths are set only within the two main columns
%\setlength{\columnseprule}{0.25pt}
\setlength{\premulticols}{1pt}
\setlength{\postmulticols}{1pt}
\setlength{\multicolsep}{1pt}
\setlength{\columnsep}{2pt}

\begin{center}
     \Large{\textbf{Computer Vision Midterm2}} \\
\end{center}

%------------------------------
\section{Stereo Vision}
The ability to infer information on the 3D structure and distance of a scene from two or more images taken from different viewpoints.
%------------------------------
%------------------------------
\subsection{A simple stereo system}
\begin{equation*}
    \frac{T+x_l-x_r}{Z-f} = \frac{T}{Z},\ Z = f\frac{T}{x_r-x_l}, \ Z = f\frac{T}{d}
\end{equation*}
\textbf{Disparity}: $d=x_r-x_l$



%------------------------------
%------------------------------
\subsection{Epipolar Geometry}
Epipoles:
\begin{itemize}
    \item  $e_l$: left image of $O_r$
    \item  $e_r$: right image of $O_l$
\end{itemize}

Epipolar plane:
\begin{itemize}
    \item Three points: $O_l$, $O_r$, and $P$ define an epipolar plane
\end{itemize}

Epipolar lines and epipolar constraint:
\begin{itemize}
    \item Intersections of epipolar plane with the image planes
    \item Corresponding points are on ``conjugate'' epipolar lines 
\end{itemize}


Find Epipoles, given $p_l$:
\begin{itemize}
    \item consider its epipolar line: $p_l, e_l$
    \item find epipolar plane: $O_l, p_l, e_l$
    \item intersect the epipolar plane with the right image plane
    \item search for $p_r$ on the right epipolar line
\end{itemize}
Parallel Cameras: 
\begin{itemize}
    \item Epipoles are at infinity
    \item Epipolar lines are parallel to the baseline
\end{itemize}


%------------------------------
%------------------------------
\subsection{Essential Matrix}

The \textbf{ESSENTIAL} matrix is a $3\times3$ matrix that “encodes” the epipolar geometry of two views. Given a point in an image, multiplying by the Essential Matrix, will tell us the \textbf{EPIPOLAR} line in the second image where the corresponding point must be.\\
\begin{equation*}
    E=RS,\ P^T_rEP_l = 0,\ p^T_rEp_l = 0 
\end{equation*}
Essential matrix has rank 2, and depends only on the EXTRINSIC Parameters ($R$ \& $T$).
\begin{equation*}
T =\left[ \begin{array}{ccc}
T_x & T_y & T_z
\end{array} \right ], S = \left[ \begin{array}{ccc}
0 & -T_z & T_y\\
T_z & 0 & -T_x\\
-T_y & T_x & 0
\end{array} \right ]
\end{equation*}
$p_r$ belongs to epipolar line in the right image defined by
\begin{equation*}
    l_r = Ep_l
\end{equation*}
$p_l$ belongs to epipolar line in the left image defined by 
\begin{equation*}
    l_l = E^Tp_r
\end{equation*}
Epipoles belong to the epipolar lines, and they belong to all the epipolar lines:
\begin{equation*}
    e^T_rE=0, Ee_l=0
\end{equation*}



%------------------------------
%------------------------------
\subsection{Fundamental Matrix}
The essential matrix uses \textbf{CAMERA} coordinates. To use image coordinates we must consider the \textbf{INTRINSIC} camera parameters.
\begin{equation*}
    p_l=M_l^{-1}\bar{p_l},\ p_r=M_r^{-1}\bar{p_r}, \bar{p_r}^TF\bar{p_l}=0,\ F=M_r^{-T}RSM^{-1}_l
\end{equation*}


Fundamental matrix has rank 2, and depends on the INTRINSIC and EXTRINSIC Parameters ($f$, etc ; $R$ \& $T$). Analogous to the Essential matrix, the Fundamental matrix also tells how points in each image are related to epipolar lines in the other image.


%------------------------------
%------------------------------
\subsection{Eight Points Algorithm}
\begin{itemize}
    \item $F$ is a $3\times3$ matrix (9 entries) but rank 2
    \item \textbf{HOMOGENEOUS} linear system with 9 unknowns
    \item Need $m \geq 8$; solution will be up to a constant
\end{itemize}
\begin{equation*}
    \left[ \begin{array}{ccc}
x_1x'_1 & ... & x_mx'_m\\
 x_1y'_1 & ... & x_my'_m \\
 x_1 & ... & x_m\\
 y_1x'_1 & ... & y_mx'_m\\
 y_1y'_1 & ... & y_my'_m\\
 y_1 & ... & y_m\\
 x'_1 & ... & x'_m\\
 y'_1 & ... & y'_m\\
 1 & ... & 1
    \end{array} \right ]^T    \left[ \begin{array}{c}
f_11\\
f_21\\
f_31\\
f_12\\
f_22\\
f_32\\
f_13\\
f_23\\
f_33
    \end{array} \right ] = 0
\end{equation*}
Assume that we need to find the non trivial solution of:
\begin{equation*}
    Ax=0
\end{equation*}
with $m$ equations and $n$ unknowns, $m \geq n – 1$ and rank$(A) = n-1$. Since the norm of $x$ is arbitrary, we will look for a solution
with norm $||x|| = 1$. We want $Ax$ as close to $0$ as possible and $||x|| =1$:
\begin{equation*}
    \min_x ||Ax||^2\ \ \text{s.t. } ||x||^2 = 1,\ ||Ax||^2 = x^TA^TAx
\end{equation*}
Define the following cost:


%------------------------------
\section{}




%------------------------------
\section{}













%------------------------------
\section{}
















%------------------------------
\section{}


\end{multicols}
\end{document}

