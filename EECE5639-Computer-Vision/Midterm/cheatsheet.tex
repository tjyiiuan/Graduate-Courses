\documentclass[10pt,landscape]{article}
\usepackage{multicol}
\usepackage{calc}
\usepackage{ifthen}
\usepackage[landscape]{geometry}
\usepackage{hyperref}
\usepackage{amsmath}
\usepackage{graphicx}
\usepackage{subfigure}
\usepackage{float}
\usepackage{ulem}
\usepackage{bm}
\usepackage{anysize}
% To make this come out properly in landscape mode, do one of the following
% 1.
%  pdflatex latexsheet.tex
%
% 2.
%  latex latexsheet.tex
%  dvips -P pdf  -t landscape latexsheet.dvi
%  ps2pdf latexsheet.ps


% If you're reading this, be prepared for confusion.  Making this was
% a learning experience for me, and it shows.  Much of the placement
% was hacked in; if you make it better, let me know...


% 2008-04
% Changed page margin code to use the geometry package. Also added code for
% conditional page margins, depending on paper size. Thanks to Uwe Ziegenhagen
% for the suggestions.

% 2006-08
% Made changes based on suggestions from Gene Cooperman. <gene at ccs.neu.edu>


% To Do:
% \listoffigures \listoftables
% \setcounter{secnumdepth}{0}


% This sets page margins to .5 inch if using letter paper, and to 1cm
% if using A4 paper. (This probably isn't strictly necessary.)
% If using another size paper, use default 1cm margins.
\ifthenelse{\lengthtest { \paperwidth = 11in}}
	{ \geometry{top=.5in,left=.5in,right=.5in,bottom=.5in} }
	{\ifthenelse{ \lengthtest{ \paperwidth = 297mm}}
		{\geometry{top=1cm,left=1cm,right=1cm,bottom=1cm} }
		{\geometry{top=1cm,left=1cm,right=1cm,bottom=1cm} }
	}

% Turn off header and footer
\pagestyle{empty}
 

% Redefine section commands to use less space
\makeatletter
\renewcommand{\section}{\@startsection{section}{1}{0mm}%
                                {-1ex plus -.5ex minus -.2ex}%
                                {0.5ex plus .2ex}%x
                                {\normalfont\large\bfseries}}
\renewcommand{\subsection}{\@startsection{subsection}{2}{0mm}%
                                {-1explus -.5ex minus -.2ex}%
                                {0.5ex plus .2ex}%
                                {\normalfont\normalsize\bfseries}}
\renewcommand{\subsubsection}{\@startsection{subsubsection}{3}{0mm}%
                                {-1ex plus -.5ex minus -.2ex}%
                                {1ex plus .2ex}%
                                {\normalfont\small\bfseries}}
\makeatother

% Define BibTeX command
\def\BibTeX{{\rm B\kern-.05em{\sc i\kern-.025em b}\kern-.08em
    T\kern-.1667em\lower.7ex\hbox{E}\kern-.125emX}}

% Don't print section numbers
\setcounter{secnumdepth}{0}


\setlength{\parindent}{0pt}
\setlength{\parskip}{0pt plus 0.5ex}


% -----------------------------------------------------------------------

\begin{document}

\raggedright
\footnotesize
\begin{multicols}{3}


% multicol parameters
% These lengths are set only within the two main columns
%\setlength{\columnseprule}{0.25pt}
\setlength{\premulticols}{1pt}
\setlength{\postmulticols}{1pt}
\setlength{\multicolsep}{1pt}
\setlength{\columnsep}{2pt}

\begin{center}
     \Large{\textbf{Computer Vision Midterm}} \\
\end{center}

\section{Pinhole Camera Model}
\subsection{Projection Equation}  $x = f X/Z$, $y = f X/Z$, $z = f$\\
The $2D$ perspective images of $3D$ parallel lines that are not parallel to the image plane are not parallel and intersect at the ``vanishing point''.\\
\subsection{3D Line Equations}
$P(X_P, Y_P, Z_P), Q(X_Q, Y_Q, Z_Q), R(X, Y, Z)$\\
$R=P+\lambda(Q-P)$\\
$X=X_P + \lambda(X_Q - X_P)$\\
$Y=Y_P + \lambda(Y_Q - Y_P)$\\
$Z=Z_P + \lambda(Z_Q - Z_P)$\\
$x=f[X_P + \lambda(X_Q - X_P)]/[Z_P + \lambda(Z_Q - Z_P)]$\\
$y=f[Y_P + \lambda(Y_Q - Y_P)]/[Z_P + \lambda(Z_Q - Z_P)]$\\
$z=f$\\
Assuming that $Z_Q \neq Z_P$:\\
$\lim\limits_{\lambda\to\infty}x \to f\frac{X_Q - X_P}{Z_Q - Z_P}$, $\lim\limits_{\lambda\to\infty}y \to f\frac{Y_Q - Y_P}{Z_Q - Z_P}$, $\lim\limits_{\lambda\to\infty}z \to f$\\
When $Z_Q == Z_P:$\\
$\lim\limits_{\lambda\to\infty}x \to \infty$, $\lim\limits_{\lambda\to\infty}y \to \infty$, $\lim\limits_{\lambda\to\infty}z \to f$\\
The image of the IDEAL point of a line parallel to the
image plane is an IDEAL point.\\
The image of a circle is the intersection of a cone and
the image plane and it is in general an ellipse.\\

\section{2D Transformation}
%------------
\subsection{translation}
\begin{equation*}
\left[ \begin{array}{c}
x'\\
y'\\
1
\end{array} \right ] = \left[ \begin{array}{ccc}
1&0&t_x\\
0&1&t_y\\
0&0&1
\end{array} \right ]\left[ \begin{array}{c}
x\\
y\\
1
\end{array} \right ] 
\end{equation*}
$\#D.O.F \ : 2$
Invariant:\\
Preserves: orientation + ...
%------------
\subsection{Rigid(Euclidean)}
\begin{equation*}
\left[ \begin{array}{c}
x'\\
y'\\
1
\end{array} \right ] = \left[ \begin{array}{ccc}
cos\theta & -sin\theta&t_x\\
sin\theta&cos\theta&t_y\\
0&0&1
\end{array} \right ] \left[ \begin{array}{c}
x\\
y\\
1
\end{array} \right ] 
\end{equation*}
$\#D.O.F \ : 3$\\
Invariant: lengths, areas\\
Preserves: lengths + ...
%------------
\subsection{Similarity(scaled Euclidean)}

\begin{equation*}
\left[ \begin{array}{c}
p'\\
1
\end{array} \right ] = \left[ \begin{array}{cc}
sR&t\\
0&1
\end{array} \right ]\left[ \begin{array}{c}
p\\
1
\end{array} \right ] 
\end{equation*}
$\#D.O.F \ : 4$\\
Invariant: Ratios of lengths, angles\\
Preserves: angles + ...
%------------
\subsection{affine}

\begin{equation*}
\left[ \begin{array}{c}
p'\\
1
\end{array} \right ] = \left[ \begin{array}{cc}
A&b\\
0&1
\end{array} \right ] \left[ \begin{array}{c}
p\\
1
\end{array} \right ] 
\end{equation*}
$\#D.O.F \ : 6$\\
Invariant: Parallellism, ratio of areas, ratio of lengths on parallel lines (e.g midpoints), linear combinations of vectors (centroids). \\
Preserves: parallelism + ...
%------------
\subsection{projective}
A projective transformation of Pn onto itself is completely determined by its action on n+2 points.\\
\begin{equation*}
\left[ \begin{array}{c}
p'\\
1
\end{array} \right ] \sim \left[ \begin{array}{cc}
A&b\\
c^T&1
\end{array} \right ] \left[ \begin{array}{c}
p\\
1
\end{array} \right ] 
\end{equation*}
$\#D.O.F \ : 8$\\
Invariant: Concurrency, collinearity, cross ratio\\
Preserves: straight lines
%------------
%\subsection{Summary of 2D Transformation}
\section{Coordinate Systems}
$P_C = M_{ext}\times P_w$, $M_{ext} = R_{(X)}R_{(Y)}R_{(Z)}T$
%------------
\subsection{3D Rotation}
\textbf{CLOCKWISE} Rotation around the coordinate axes (left hand):
\begin{equation*}
\left[ \begin{array}{c}
X'\\
Y'\\
Z'
\end{array} \right ] = R*\left[ \begin{array}{c}
X\\
Y\\
Z
\end{array} \right ]
\end{equation*}

\begin{equation*}
R_x(\alpha) = \left[ \begin{array}{ccc}
1&0&0\\
0&cos\alpha & -sin\alpha\\
0 & sin\alpha&cos\alpha
\end{array} \right ]
\end{equation*}

\begin{equation*}
R_y(\beta) = \left[ \begin{array}{ccc}
cos\beta&0&sin\beta\\
0&1&0\\
-sin\beta&0&cos\beta
\end{array} \right ]
\end{equation*}

\begin{equation*}
R_z(\gamma) = \left[ \begin{array}{ccc}
cos\gamma&-sin\gamma&0\\
sin\gamma&cos\gamma&0\\
0&0&1
\end{array} \right ]
\end{equation*}
%------------
\subsection{Translation}
\begin{equation*}
T = \left[ \begin{array}{cccc}
1&0&0&t_x\\
0&1&0&t_y\\
0&0&1&t_z\\
0&0&0&1
\end{array} \right ]
\end{equation*}

%------------
%------------



\section{RGB Model}
%------------
\subsection{CMY}
\begin{equation*}
\left[ \begin{array}{c}
C\\
M\\
Y
\end{array} \right ] = 1 - \left[ \begin{array}{c}
R\\
G\\
B
\end{array} \right ] 
\end{equation*}
%------------
\subsection{YIQ}
\begin{equation*}
\left[ \begin{array}{c}
Y\\
I\\
Q
\end{array} 
\right] = {\left[ \begin{array}{ccc}
0.299 & 0.587 & 0.114\\
0.596 & -0.275 & -0.321\\
0.212 & -0.532 & 0.311
\end{array} 
\right]}  
\left[ \begin{array}{c}
R\\
G\\
B
\end{array} 
\right]
\end{equation*}
%------------
%------------
\subsection{HSI}

\begin{equation*}
\theta = cos^{-1}\left(\frac{[(R-G)+(R-B)]/2}{[(R-G)^2+(R-B)(G-B)]^{1/2}}\right)
\end{equation*}

\begin{equation*}
H = 360 - \theta\ if\ B\ >\ G\ else\ \theta
\end{equation*}
$S = 1-\frac{3}{R+G+B}[min(R, G, B)]$, $I = \frac{R+G+B}{3}$

%------------
%------------

\section{Image processing}

%------------
\subsection{Spatial Filtering}
Low-pass filters eliminate/attenuate high frequency components in the frequency domain (sharp image details), and result in image blurring.\\
High-pass filters attenuate/eliminate low-frequency components (resulting in sharpening edges and other sharp details).\\
Band-pass filters remove selected frequency regions
between low and high frequencies (for image restoration, not enhancement).\\

%------------
\subsection{Noise}
Mean $\mu = \frac{1}{N}\sum^N_{i=1}x_i$\\
Variance $\sigma^2 = \frac{1}{N-1} \sum^N_{i=1}(x_i - \mu)^2$\\
Gaussian $N(\mu, \sigma^2) \sim \frac{1}{\sqrt{2\pi}\sigma}e^{-\frac{(x-\mu)^2}{x\sigma^2}}$\\
Additive noise $I(i,j) = \Hat{I}(i,j) + N(i,j)$\\
Multiplicative noise  $I(i,j) = \Hat{I}(i,j) \times N(i,j)$\\
Impulsive noise (salt and pepper): 
\begin{equation*}  
I(i,j) = \left\{  
     \begin{array}{cc}  
     \Hat{I}(i,j), & x < l\\  
     i_{min} + y(i_{max} - i_{min}), & x \leq l\\   
     \end{array}  
\right.  
\end{equation*}  

%------------
\subsection{Smoothing Filters}
They are used for blurring and noise reduction. Blurring is performed as pre-processing to remove small detail or bridge curve gaps. Noise reduction can be done by linear or nonlinear filtering.\\
Linear smoothing Filters are simply averaging filters. They compute the average of the filters under the mask. They reduce sharp transitions. They are low pass filters.\\

\subsection{Linear Filtering}
\begin{equation*}
\text{BOX Filter: \   } \frac{1}{9}\left[ \begin{array}{ccc}
1 & 1 & 1\\
1 & 1 & 1\\
1 & 1 & 1
\end{array} 
\right]
\end{equation*}

Reduce Noise:\\
Expected value of a pixel $E[I_i] = \hat{I}_i \to \frac{1}{mn}\sum_i^{mn}\hat{I}_i$\\
Variance of a pixel $E[(I_i - E[I_i])^2] = \sigma^2 \to \frac{\sigma^2}{mn}$\\
The bigger the mask, more neighbors contribute; smaller noise variance of the output; bigger noise spread; more blurring; more expensive to compute.

\begin{equation*}
    \text{Weighted Average Filter: } w(s,t) = Ke^{\frac{s^2 + t^2}{2\sigma^2}}
\end{equation*}
Gives more weight at the central pixel and less weights to the neighbors. The farther away the neighbors, the smaller the weight. Less blurring of edges.\\
Both, the BOX filter and the Gaussian filter are
separable. First convolve each row with a 1D filter. Then convolve each column with a 1D filter.
%------------
\subsection{Nonlinear Filtering}
Limitation of averaging: Signal frequencies shared with noise are lost, resulting in blurring. Impulsive noise is diffused but not removed. It spreads pixel values, resulting in blurring.\\
Replace each pixel with the \textbf{MEDIAN} value of all the pixels in the neighborhood.\\
Properties: Non-linear; Does not spread the noise
Can remove spike noise; Expensive to run.


\section{Image Features}
%------------
\subsection{Edge Detection}

\subsubsection{Prewitt}
\begin{equation*}
\text{vertical: } \left[ \begin{array}{ccc}
-1 & 0 & 1\\
-1 & 0 & 1\\
-1 & 0 & 1
\end{array}  
\right]
\text{horizontal: } \left[ \begin{array}{ccc}
-1 & -1 & -1\\
0 & 0 & 0\\
1 & 1 & 1
\end{array} \right]
\end{equation*}

\subsubsection{Sobel}
\begin{equation*}
\text{vertical: } \left[ \begin{array}{ccc}
-1 & 0 & 1\\
-2 & 0 & 2\\
-1 & 0 & 1
\end{array}  
\right]
\text{horizontal: } \left[ \begin{array}{ccc}
-1 & -2 & -1\\
0 & 0 & 0\\
1 & 2 & 1
\end{array} \right]
\end{equation*}

\begin{equation*}
\text{Magnitude  } G = \sqrt{G_x^2 + G_y^2}
\end{equation*}
\begin{equation*}
\text{Orientation  } \Theta = \arctan(\frac{G_y}{G_x})
\end{equation*}

\subsubsection{CANNY ENHANCER}
The input is image $I$; $G$ is a zero mean Gaussian filter (std = $\sigma$)\\
1. $J = I \times G$ (smoothing)\\
2. For each pixel $(i,j)$: (edge enhancement)\\
Compute the image gradient $\nabla J(i,j) = (Jx(i,j),Jy(i,j))’$ \\
Estimate edge strength
$E_s(i,j) = \sqrt{(J_x^2(i,j) + J_y^2(i,j))}$\\
Estimate edge orientation
$E_o(i,j) = arctan(J_x(i,j)/J_y(i,j))$\\
The output are images $E_s$ and $E_o$


\subsubsection{NONMAX SUPRESSION}
The inputs are $E_s$ \& $E_o$ (outputs of CANNY\_ENHANCER)\\
Consider 4 directions $D = \{0, 45, 90, 135\}$ w.r.t. x\\
For each pixel $(i,j)$ do:\\
Find the direction $d\in D$ s.t. $d\sim E_o(i,j)$ (normal to the edge)\\
If $E_s(i,j)$ is smaller than at least one of its neigh. along d\\
$I_N(i,j)=0$\\
Otherwise, $I_N(i,j)= E_s(i,j)$\\
The output is the thinned edge image $I_N$

\subsubsection{Thresholding}
Edges are found by thresholding the output of NONMAX SUPRESSION. If the threshold is too high: Very few (none) edges High MISDETECTIONS, many gaps. If the threshold is too low: Too many (all pixels) edges High FALSE POSITIVES, many extra edges.

\subsection{Corner}

\subsubsection{Harris Corner Detection}
Compute the Image Gradient\\
$I_x = G_{x,s} \times I$, $Iy = G_{y,s} \times I$\\
Compute products of derivatives at each pixel\\
$I^2_x = I_x \cdot I_x$, $I^2_y = I_y \cdot I_y$ and $I_{xy} = I_x \cdot I_y$\\
Compute the sums of the products at each pixel using a window averaging\\
$S^2_x = Gs’ \times I^2_x$, $S^2_y = Gs’ \times I^2_y$, $S_{xy} = Gs’ \times I_{xy}$\\
Define the Matrix at each pixel $M = [S^2_x\ \  S_{xy} ; S_{xy}\ \ S^2_y]$\\
Compute the response $R = det(M) - k\ trace(M)^2$\\
Threshold R \\
Compute Nonmax suppression

\section{Grouping edges}
Edge detectors find “edges” (pixel level)
To perform image analysis: edges must be grouped into entities such as contours (higher level). Canny does this to certain extent: the detector finds chains of edges.
%------------
\subsection{Hough Transform}
Given a set of collinear edge points, each of them have associated a line in parameter space. These lines intersect at the point (m,n) corresponding to the parameters of the line in the image space. At each point of the (discrete) parameter space, count how many lines pass through it. Use an array of counters.Can be thought as a “parameter image”. The higher the count, the more edges are collinear in the image
space. Find a peak in the counter array
This is a “bright” point in the parameter image. It can be found by thresholding.

\subsubsection{Normal equation}
\begin{equation*}
    \rho = x\ cos\theta + y\ sin\theta \text{\ \ \ \ $x,\ y$ on the line}
\end{equation*}

\begin{equation*}
\left[ \begin{array}{cc}
\rho<0\ \  \theta  < 0 & \rho>0\ \  \theta > 0\\
\rho<0\ \  \theta > 0 & \rho>0\ \  \theta < 0
\end{array} \right]-\frac{D}{2} \leq \rho \leq \frac{D}{2}\ -90\leq\theta\leq90
\end{equation*}
\begin{equation*}
    y=k \to \rho = k, \theta = 90 \ \ \ \ \ \     x=k \to \rho = k, \theta = 0
\end{equation*}

\subsection{Mean Shift} 
1. Choose a window size.\\
2. Choose the initial location of the search window.\\
3. Compute the mean location in the search window.\\
4. Center the window at the location computed in 3.\\
5. Repeat steps 3 and 4 until convergence.

\subsection{Merging Algorithm}
Consider two regions:\\
Let $g_1$, $g_2$, $\dots$, $g_{m1}$ be the gray values in $R1$\\
Let $g_{m1+1}$, $g_{m1+2}$, $\dots$, $g_{m1+m2}$ be the gray values in $R2$\\
Assume regions are constant with uncorrelated zero mean Gaussian noise.\\
Two possible HYPOTHESES:\\
\textbf{$H_0$}: Both regions belong to the same object and the residual values have distribution $N(0, \sigma_0^2)$.\\
\textbf{$H_1$}: Each region belongs to a different object and their residual values have distributions $N(0, \sigma_1^2)$ and $N(0, \sigma_2^2)$.\\
Assume $H_0$ is \textbf{TRUE}: Estimate $\mu_0$ and $\sigma_0$:
\begin{equation*}
    \mu_0 = \frac{1}{m_1+m_2}\sum_{g_i\in R_1\cup R_2}g_i,\ \ \sigma^2_0 = \frac{1}{m_1+m_2}\sum_{g_i\in R_1\cup R_2}(g_i - \mu_0)^2
\end{equation*}
Compute the probability of independently drawing $g_1$, $g_2$, $\dots$, $g_{m1+m_2}$ with distribution $N(\mu_0,\sigma_0^2)$:
\begin{equation*}
\begin{aligned}
P\left ( g_1, g_2,...,g_{m_1+m_2}|H_0 \right )&=\prod^{m_1+m_2}_{i=1}P(g_i|H_0)\\
&=\frac{1}{(\sqrt{2\pi}\sigma_0)^{m_1+m_2}}e^{-(m_1+m_2)/2}
\end{aligned}
\end{equation*}

Assume $H_1$ is \textbf{TRUE}: Estimate $\mu_1$ and $\sigma_1$, $\mu_2$ and $\sigma_2$:
\begin{equation*}
    \mu_j = \frac{1}{m_j}\sum_{g_i\in R_j}g_i,\ \ \sigma^2_0 = \frac{1}{m_j}\sum_{g_i\in R_j}(g_i - \mu_j)^2, \ \ j = 1,2
\end{equation*}
Compute the probability of independently drawing $g_1$, $g_2$, $\dots$, $g_{m1}$ with distribution $N(\mu_1,\sigma_1^2)$, and $g_{m1+1}$, $g_{m1+2}$, $\dots$, $g_{m1+m2}$ with
distribution $N(\mu_2,\sigma_2^2)$:
\begin{equation*}
P\left (g_1,...,g_{m1}|H_1 \right )
=\frac{1}{(\sqrt{2\pi}\sigma_1)^m_1}e^{-m_1/2}
\end{equation*}

\begin{equation*}
P\left (g_{m_1 +1},...,g_{m1_m2}|H_1 \right )
=\frac{1}{(\sqrt{2\pi}\sigma_2)^m_2}e^{-m_2/2}
\end{equation*}

\begin{equation*}
P\left (g_1, g_2,...,g_{m_1+m_2}|H_1 \right )=\frac{1}{(\sqrt{2\pi})^{m_1+m_2}\sigma_1^{m1}\sigma_2^{m2}}e^{-(m_1+m_2)/2}
\end{equation*}

\begin{equation*}
L = \frac{P\left (g_1, g_2,...,g_{m_1+m_2}|H_1 \right )}{P\left (g_1, g_2,...,g_{m_1+m_2}|H_0 \right )}=\frac{\sigma_0^{m+n}}{\sigma_1^m\sigma_2^n}
\end{equation*}
If $L < 1$, $H_0$ is more likely than $H_1$: Merge the regions!

\subsection{Feature Matching}
\begin{equation*}
    SSD = \sum_{[i,j] \in R}(f(i,j) - g(i,j))^2
\end{equation*}

\begin{equation*}
   c_{fg}= \sum_{[i,j] \in R}f(i,j)g(i,j)
\end{equation*}
\begin{equation*}
N_{fg} = \sum_{[i, j]\in R}\hat{f}(i, j)\hat{g}(i, j),\ \ \hat{f} = \frac{f}{||f||}, \ \  \hat{g} = \frac{g}{||g||}
\end{equation*}


\subsection{SIFT + RANSAC for Homography}
Choose 2 overlapping images.\\
Find SIFT features for each image.\\
Match SIFT features to get initial point correspondences.\\
Run RANSAC:\\
1. Select minimal number of points (4), find homography.\\
2. Check number of data points consistent with this fit.\\
3. Good enough?  $\to$ Find homography using all inliers.\\


\subsection{Homograph Estimation}
\begin{equation*}
\left[ \begin{array}{c}
x'\\
y'\\
1
\end{array} \right] = \left[ \begin{array}{ccc}
h_{11} & h_{12} & h_{13} \\
h_{21} & h_{22} & h_{23} \\
h_{31} & h_{32} & h_{33} 
\end{array} \right]\left[ \begin{array}{c}
x\\
y\\
1
\end{array} \right] 
\end{equation*}

\begin{equation*}
\left[ \begin{array}{cccccccc}
x_1 & y_1 & 1 & 0 & 0 & 0 & -x_1x_1'& - y_1x_1'\\
0 & 0 & 0 & x_1 & y_1 & 1 & -x1y_1'& -y_1y_1'\\
...
\end{array} \right] 
\end{equation*}
\begin{equation*}
Ah=b \to h = (A^TA)^{-1}(A^Tb)
\end{equation*}




\subsection{Image (un)Warping}
$I = (1-a)\ (1-b)\ I(i,j)+a\ (1-b)\ I(i+1,j)$\\
 \ \   $+(1-a)\ b\ I(i,j+1)+a\ b\ I(i+1, j+1)$



\end{multicols}
\end{document}

\section{}
\subsection{}