\documentclass[12pt]{article}

\usepackage{amsmath}
\usepackage{graphicx}
\usepackage{subfigure}
\usepackage{float}
\usepackage{ulem}
\usepackage{bm}
\usepackage{anysize}
\usepackage{pythonhighlight}

\marginsize{2cm}{2cm}{0.9cm}{1.8cm}

\title{EECE 5639 Computer Vision\\ [2ex] \begin{large} Homework \#6 \end{large} }
\author{Jiyu Tian}
\date{}

\begin{document}
\maketitle
\pagestyle{empty}
%%---------------------------------------------------------------
%% Question 1
%%---------------------------------------------------------------
\section{Solution:}
The planar homography relates the transformation between two views when the scene is planar or the two views were generated by a camera rotating around its center of projection. It is a $3\times3$ matrix of rank 3. It has no information about the scene depth.\\
\\
The Essential and Fundamental matrices completely describe the geometric relationship between corresponding points of a stereo pair. They are $3\times3$ matrices of rank 2. They provide the epipolar line of second image to search along once given a point in first image.
%%---------------------------------------------------------------
%% Question 2
%%---------------------------------------------------------------
\section{Solution:}
If two cameras have parallel optical axes and the line through the two centers of projection is perpendicular to the optic axes, the corresponding epipoles are at infinity.

%%---------------------------------------------------------------
%% Question 3
%%---------------------------------------------------------------
\section{Solution:}
Projection of planar points to image plane is an affine transformation, if the planar surface is perpendicular to the optic axis:
\begin{equation*}
\begin{aligned}
\left[ \begin{array}{c}
x\\
y\\
1
\end{array} \right] &\sim \left[ \begin{array}{cccc}
f & 0 & 0 & 0\\
0 & f & 0 & 0\\
0 & 0 & 1 & 0
\end{array} \right]\left[ \begin{array}{cccc}
r_{11} & r_{12} & r_{13} & t_x \\
r_{21} & r_{22} & r_{23} & t_y\\
r_{31} & r_{32} & r_{33} & t_z\\
0 & 0 & 0 & 1
\end{array} \right]\left[ \begin{array}{c}
X\\
Y\\
0\\
1
\end{array} \right] \\
&\sim \left[ \begin{array}{cccc}
f & 0 & 0\\
0 & f & 0\\
0 & 0 & 1
\end{array} \right]\left[ \begin{array}{cccc}
r_{11} & r_{12} & t_x \\
r_{21} & r_{22} & t_y\\
r_{31} & r_{32} & t_z\\
\end{array} \right]\left[ \begin{array}{c}
X\\
Y\\
1
\end{array} \right] \\
&\sim \left[ \begin{array}{cccc}
fr_{11} & fr_{12} & ft_x \\
fr_{21} & fr_{22} & ft_y\\
r_{31} & r_{32} & t_z\\
\end{array} \right]\left[ \begin{array}{c}
X\\
Y\\
1
\end{array} \right]\\
&\sim \frac{1}{t_z} \left[ \begin{array}{cccc}
f\cos\theta & -f\sin\theta &  ft_x \\
f\sin\theta & f\cos\theta & ft_y\\
0 & 0 & t_z\\
\end{array} \right]\left[ \begin{array}{c}
X\\
Y\\
1
\end{array} \right]
\end{aligned}
\end{equation*}
It is also a 2D affine transformation from film coordinates $(x,y)$ to pixel coordinates $(u,v)$:
\begin{equation*}
\left[ \begin{array}{c}
u\\
v\\
1
\end{array} \right] \sim \left[ \begin{array}{cccc}
a_{11} & a_{12} & a_{13} \\
a_{21} & a_{22} & a_{23}\\
0 & 0 & 1\\
\end{array} \right]\left[ \begin{array}{c}
x\\
y\\
1
\end{array} \right]
\end{equation*}
Since the combination of affine transformations is also an affine transformation, the projection of a point in a planar scene at world coordinates $(X, Y)$ to pixel coordinates
$(u, v)$ in an image plane can be represented using a \textit{planar affine camera model}.\\
\\
The affine viewing condition is small depth variation in the scene along the optic axis compared to its average distance from the camera. There are 6 $D.O.F.$ in the model. Minimum of 3 calibration points are needed to estimate the transformation. When an affine transform is applied to two parallel lines, the transformed lines will also be parallel.
%%---------------------------------------------------------------
%% Question 4
%%---------------------------------------------------------------
\section{Solution:}
Let $O_{i1}$, $O_{i2}$ be the origin of image plane 1 and 2 respectively, we have $O_{i1}(0,9)$, $O_{i1}(0, -9)$.
\begin{equation*}  
\begin{aligned}
\left.  
     \begin{array}{cc}  
        \text{Line } PO_1: y = \frac{1}{5}x + 12\\
        \text{Line } p_1O_{i1}: y = 10x + 9\\  
     \end{array}  
\right\}&\to  p_1(\frac{15}{49}, \frac{591}{49})\\
\left.  
     \begin{array}{cc}  
        \text{Line } PO_2: y = \frac{3}{5}x -4 \\
        \text{Line } p_2O_{i2}: y = -10x - 9\\  
     \end{array}  
\right\}&\to  p_2(-\frac{25}{53}, -\frac{227}{53})
\end{aligned}
\end{equation*}  

Disparity
\begin{equation*}
\begin{aligned}
    p_1O_{i1} &= \sqrt{(\frac{15}{49})^2 + (\frac{591}{49} - 9)^2} = \frac{15}{49}\sqrt{101}\\
    p_2O_{i2} &= \sqrt{(-\frac{25}{53})^2 + (-\frac{227}{53} + 9)^2} = \frac{25}{53}\sqrt{101}\\
    d &= p_1O_{i1} - p_2O_{i2} = \frac{430}{2597}\sqrt{101}
\end{aligned}
\end{equation*}

%%---------------------------------------------------------------
%% Question 5
%%---------------------------------------------------------------
\section{Solution:}
Given a set of points $\textbf{p}_i = [x_i, y_i, 1]^T$ with $i = 1, ..., n$ from either side, we can define
\begin{equation*}
    \bar{x} = \frac{1}{n}\sum_{i=1}^nx_i,\ \bar{y} = \frac{1}{n}\sum_{i=1}^ny_i,\ \bar{d} = \frac{\sum_{i=1}^n\sqrt{(x_i - \bar{x})^2 + (y_i - \bar{y})^2 }}{n\sqrt{2}}
\end{equation*}
Solve $H\textbf{p}_i = \hat{\textbf{p}}_i$ for a $3\times3$ matrix $H$ with $\hat{\textbf{p}}_i = [(x_i - \bar{x})/d,\ (y_i - \bar{y})/d,\ i]^T$:
\begin{equation*}
H = \left[ \begin{array}{ccc}
1/d & 0 & -\bar{x}/d \\
0 & 1/d & -\bar{y}/d \\
0 & 0 & 1
\end{array} \right]
\end{equation*}



%%---------------------------------------------------------------
%% Question 6
%%---------------------------------------------------------------
\section{Solution:}
The planar affine transformation is given by
\begin{equation*}
\left[ \begin{array}{c}
x'\\
y'\\
1
\end{array} \right] = \left[ \begin{array}{ccc}
h_{11} & h_{12} & h_{13} \\
h_{21} & h_{22} & h_{23} \\
0 & 0 & 1
\end{array} \right]\left[ \begin{array}{c}
x\\
y\\
1
\end{array} \right] 
\end{equation*}
Since there are 6 $D.O.F.$, 3 points pairs are needed for solution. Given the map relation from $(x_1, y_1)$, $(x_2, y_2)$, $(x_3, y_3)$ to $(x'_1, y'_1)$, $(x'_2, y'_2)$, $(x'_3, y'_3)$, we obtain

\begin{equation*}
    Ax = b
\end{equation*}
where
\begin{equation*}
A = \left[ \begin{array}{cccccc}
x_1 & y_1 & 1 & 0 & 0 & 0\\
0 & 0 & 0 & x_1 & y_1 & 1\\
x_2 & y_2 & 1 & 0 & 0 & 0\\
0 & 0 & 0 & x_2 & y_2 & 1\\
x_3 & y_3 & 1 & 0 & 0 & 0\\
0 & 0 & 0 & x_3 & y_3 & 1
\end{array} \right],\  x= \left[ \begin{array}{c}
h_{11} \\
h_{12} \\
h_{13} \\
h_{21} \\
h_{22} \\
h_{23} 
\end{array} \right],\  b= \left[ \begin{array}{c}
x'_1\\
y'_1\\
x'_2\\
y'_2\\
x'_3\\
y'_3
\end{array} \right] 
\end{equation*}
To solve via least squares 
\begin{equation*}
    X = \arg\max_{x}||Ax-b||^2 = (A^TA)^{-1}A^Tb
\end{equation*}
Here we can separate the $6\times6$ system into $2$ sub-equations:
\begin{equation*}
A_1x_1 = b_1, A_1 = \left[ \begin{array}{ccc}
x_1 & y_1 & 1\\
x_2 & y_2 & 1\\
x_3 & y_3 & 1
\end{array} \right],\  x_1= \left[ \begin{array}{c}
h_{11} \\
h_{12} \\
h_{13}
\end{array} \right],\  b_1= \left[ \begin{array}{c}
x'_1\\
x'_2\\
x'_3
\end{array} \right] 
\end{equation*}
\begin{equation*}
A_2x_2 = b_2, A_2 = \left[ \begin{array}{cccccc}
x_1 & y_1 & 1\\
x_2 & y_2 & 1\\
x_3 & y_3 & 1
\end{array} \right],\  x_2 = \left[ \begin{array}{c}
h_{21} \\
h_{22} \\
h_{23} 
\end{array} \right],\  b_2 = \left[ \begin{array}{c}
y'_1\\
y'_2\\
y'_3
\end{array} \right] 
\end{equation*}
which can be solved via
\begin{equation*}
    X_1 = \arg\max_{x_1}||A_1x_1-b_1||^2 = (A_1^TA_1)^{-1}A_1^Tb_1
\end{equation*}
\begin{equation*}
    X_2 = \arg\max_{x_2}||A_2x_2-b_2||^2 = (A_2^TA_2)^{-1}A_2^Tb_2
\end{equation*}

\begin{equation*}
X = \left[ \begin{array}{c}
X_1\\
X_2
\end{array} \right]
\end{equation*}
%%---------------------------------------------------------------
%% Question 7
%%---------------------------------------------------------------
\section{Solution:}
\begin{equation*}
T_1 = \left[ \begin{array}{cccc}
1 & 0 & 0 & -2\\
0 & 1 & 0 & -2\\
0 & 0 & 1 & -4\\
0 & 0 & 0 & 1
\end{array} \right],\ T_2 = \left[ \begin{array}{cccc}
1 & 0 & 0 & -4\\
0 & 1 & 0 & -3\\
0 & 0 & 1 & -3\\
0 & 0 & 0 & 1
\end{array} \right]
\end{equation*}


\begin{equation*}
R_{Z90}  = \left[ \begin{array}{cccc}
0 & -1 & 0 & 0\\
1 & 0 & 0 & 0\\
0 & 0 & 1 & 0\\
0 & 0 & 0 & 1\\
\end{array} \right],\ R_{Y180}  = \left[ \begin{array}{cccc}
-1& 0 & 0 & 0\\
0 & 1 & 0 & 0\\
0 & 0 & -1& 0\\
0 & 0 & 0 & 1
\end{array} \right]
\end{equation*}

\begin{equation*}
M_1 = R_{Y180} R_{Z90} T_1 = \left[ \begin{array}{cccc}
0 & 1 & 0 & -2\\
1 & 0 & 0 & -2\\
0 & 0 & -1 & 4\\
0 & 0 & 0 & 1
\end{array} \right]
\end{equation*}

\begin{equation*}
M_2 = R_{Y180} R_{Z90} T_2 = \left[ \begin{array}{cccc}
0 & 1 & 0 & -3\\
1 & 0 & 0 & -4\\
0 & 0 & -1 & 3\\
0 & 0 & 0 & 1
\end{array} \right]
\end{equation*}

\noindent (a) 
\begin{equation*}
C_2^1 = M_1C_2 = \left[ \begin{array}{cccc}
0 & 1 & 0 & -2\\
1 & 0 & 0 & -2\\
0 & 0 & -1 & 4\\
0 & 0 & 0 & 1
\end{array} \right] \left[ \begin{array}{c}
4\\
3\\
3\\
1
\end{array} \right] = \left[ \begin{array}{c}
1\\
2\\
1\\
1
\end{array} \right], \ E_1^1 = \left[ \begin{array}{c}
f\frac{x}{z}\\
f\frac{y}{z}\\
f
\end{array} \right]  = \left[ \begin{array}{c}
1\\
2\\
1
\end{array} \right]
\end{equation*}


\begin{equation*}
C_1^2 = M_2C_1 = \left[ \begin{array}{cccc}
0 & 1 & 0 & -3\\
1 & 0 & 0 & -4\\
0 & 0 & -1 & 3\\
0 & 0 & 0 & 1
\end{array} \right] \left[ \begin{array}{c}
2\\
2\\
4\\
1
\end{array} \right] = \left[ \begin{array}{c}
-1\\
-2\\
-1\\
1
\end{array} \right], \ E_2^2 = \left[ \begin{array}{c}
f\frac{x}{z}\\
f\frac{y}{z}\\
f
\end{array} \right]  = \left[ \begin{array}{c}
1\\
2\\
1
\end{array} \right]
\end{equation*}


\noindent (b) The camera 1 coordinates of fly are 
\begin{equation*}
f^1 =  \left[ \begin{array}{c}
0\\
2k\\
k\\
1
\end{array} \right]
\end{equation*}
The camera 2 coordinates of fly are 

\begin{equation*}
f^2 = M_2M_1^{-1}f^1 = \left[ \begin{array}{cccc}
1 & 0 & 0 & -1\\
0 & 1 & 0 & -2\\
0 & 0 & 1 & -1\\
0 & 0 & 0 & 1
\end{array} \right] \left[ \begin{array}{c}
0\\
2k\\
k\\
1
\end{array} \right] = \left[ \begin{array}{c}
-1\\
2k-2\\
k-1\\
1
\end{array} \right] 
\end{equation*}
The camera 2 coordinates of the image in the second camera
are 
\begin{equation*}
f_2^2 = \left[ \begin{array}{c}
\frac{1}{1-k}\\
2\\
1
\end{array} \right] 
\end{equation*}
Note that this is the epipolar line of fly in camera 2 coordinates

\begin{equation*}  
\left\{  
     \begin{array}{cc}  
    y = 2\\
    z = 1
     \end{array}  
\right.  
\end{equation*}  
The epipolar plane also passes through origin $(0,0,0)$, and thus we have its expression:
\begin{equation*}  
y-2z = 0
\end{equation*} 

\noindent (c) The camera 1 coordinates of fly at $t$ are  
\begin{equation*}
f_{1t}^1 = M_1 \left ( M_1^{-1}f^1 + vt \right ) = \left[ \begin{array}{cccc}
0 & 1 & 0 & -2\\
1 & 0 & 0 & -2\\
0 & 0 & -1 & 4\\
0 & 0 & 0 & 1
\end{array} \right] \left[ \begin{array}{c}
2 + 2k - 3t\\
2 - 2t\\
4 - k - t\\
1
\end{array} \right] = \left[ \begin{array}{c}
-2t\\
2k-3t\\
k+t\\
1
\end{array} \right]
\end{equation*}
The camera 1 coordinates of its image are 
\begin{equation*}
p_{1t}^1 = \left[ \begin{array}{c}
f\frac{x}{z}\\
f\frac{y}{z}\\
f
\end{array} \right]  = \left[ \begin{array}{c}
\frac{-2t}{k+t}\\
\frac{2k-3t}{k+t}\\
1
\end{array} \right]
\end{equation*}
FOE is 
\begin{equation*}
p^1_0 = \lim\limits_{t\to\infty}p_{1t}^1 = \lim\limits_{t\to\infty} \left[ \begin{array}{c}
\frac{-2t}{k+t}\\
\frac{2k-3t}{k+t}\\
1
\end{array} \right] = \left[ \begin{array}{c}
-2\\
-3\\
1
\end{array} \right]
\end{equation*}
Therefore the camera coordinates of the FOE in camera 1 are $(-2, -3, 1)$.


%%---------------------------------------------------------------
%% Question 8
%%---------------------------------------------------------------
\section{Solution:}
(a) The complexity of the underlying dynamics is $2$ since rank$(H) = 2$.\\
\\
(b) $a_1 = 2, a_2 = 1$\\
\\
(C) $x = 29$, $y = 169$.




\end{document}