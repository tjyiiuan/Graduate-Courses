\documentclass[10pt,landscape]{article}
\usepackage{multicol}
\usepackage{calc}
\usepackage{ifthen}
\usepackage[landscape]{geometry}
\usepackage{hyperref}
\usepackage{amsmath}
\usepackage{graphicx}
\usepackage{subfigure}
\usepackage{float}
\usepackage{ulem}
\usepackage{bm}
\usepackage{anysize}
\usepackage{MnSymbol}
\usepackage{listings}
\usepackage{color}
\usepackage{xcolor}
\definecolor{dkgreen}{rgb}{0,0.6,0}
\definecolor{gray}{rgb}{0.5,0.5,0.5}
\definecolor{mauve}{rgb}{0.58,0,0.82}
\lstset{frame=tb,
     language=Java,
     aboveskip=3mm,
     belowskip=3mm,
     showstringspaces=false,
     columns=flexible,
     basicstyle = \ttfamily\small,
     numbers=none,
     numberstyle=\tiny\color{gray},
     keywordstyle=\color{blue},
     commentstyle=\color{dkgreen},
     stringstyle=\color{mauve},
     breaklines=true,
     breakatwhitespace=true,
     tabsize=3
}
% To make this come out properly in landscape mode, do one of the following
% 1.
%  pdflatex latexsheet.tex
%
% 2.
%  latex latexsheet.tex
%  dvips -P pdf  -t landscape latexsheet.dvi
%  ps2pdf latexsheet.ps


% If you're reading this, be prepared for confusion.  Making this was
% a learning experience for me, and it shows.  Much of the placement
% was hacked in; if you make it better, let me know...


% 2008-04
% Changed page margin code to use the geometry package. Also added code for
% conditional page margins, depending on paper size. Thanks to Uwe Ziegenhagen
% for the suggestions.

% 2006-08
% Made changes based on suggestions from Gene Cooperman. <gene at ccs.neu.edu>


% To Do:
% \listoffigures \listoftables
% \setcounter{secnumdepth}{0}


% This sets page margins to .5 inch if using letter paper, and to 1cm
% if using A4 paper. (This probably isn't strictly necessary.)
% If using another size paper, use default 1cm margins.
\ifthenelse{\lengthtest { \paperwidth = 11in}}
	{ \geometry{top=.5in,left=.5in,right=.5in,bottom=.5in} }
	{\ifthenelse{ \lengthtest{ \paperwidth = 297mm}}
		{\geometry{top=1cm,left=1cm,right=1cm,bottom=1cm} }
		{\geometry{top=1cm,left=1cm,right=1cm,bottom=1cm} }
	}

% Turn off header and footer
\pagestyle{empty}
 

% Redefine section commands to use less space
\makeatletter
\renewcommand{\section}{\@startsection{section}{1}{0mm}%
                                {-1ex plus -.5ex minus -.2ex}%
                                {0.5ex plus .2ex}%x
                                {\normalfont\large\bfseries}}
\renewcommand{\subsection}{\@startsection{subsection}{2}{0mm}%
                                {-1explus -.5ex minus -.2ex}%
                                {0.5ex plus .2ex}%
                                {\normalfont\normalsize\bfseries}}
\renewcommand{\subsubsection}{\@startsection{subsubsection}{3}{0mm}%
                                {-1ex plus -.5ex minus -.2ex}%
                                {1ex plus .2ex}%
                                {\normalfont\small\bfseries}}
\makeatother

% Define BibTeX command
\def\BibTeX{{\rm B\kern-.05em{\sc i\kern-.025em b}\kern-.08em
    T\kern-.1667em\lower.7ex\hbox{E}\kern-.125emX}}

% Don't print section numbers
\setcounter{secnumdepth}{0}


\setlength{\parindent}{0pt}
\setlength{\parskip}{0pt plus 0.5ex}


% -----------------------------------------------------------------------

\begin{document}

\raggedright
\footnotesize
\begin{multicols}{3}


% multicol parameters
% These lengths are set only within the two main columns
%\setlength{\columnseprule}{0.25pt}
\setlength{\premulticols}{1pt}
\setlength{\postmulticols}{1pt}
\setlength{\multicolsep}{1pt}
\setlength{\columnsep}{2pt}

\begin{center}
     \Large{\textbf{Database Midterm $_{by\ Xiaoyi \ Ma}$}} \\
\end{center}
%----------------------------------------
\section{UML and Design}
%--------------------------
\subsection{Class diagrams} 

\textbf{Cardinality} denotes how many entities participate in the relationship. In general cardinality is denoted as N..M\\
* denotes zero or more participants. * $\sim$ 0..*, 1 $\sim$ 1..1\\

\textbf{Multiplicity} [m..n]. m: minimum, n: maximum, *: unlimited.\\

$<<$\textbf{enumeration}$>>$

\textbf{Association} specified with a line between the two classes.\\

Inheritance, Specialization/Generalization $\triangle$.\\

If \textbf{lifecycle} of parts depends on container use composition $\blacklozenge$; else use aggregation $\lozenge$.\\

Use \textbf{reification} to remove many-many.\\

A relationship is \textbf{redundant} if removing it does not change the information content of the class diagram.\\

%--------------------------
\subsection{Relational Algebra}
%-----------------
\subsubsection{Single Table Operators}
\textbf{Select}: returns same columns as input table, removes some rows
\textbf{Project}: returns same rows as input table, removes some columns
\textbf{Sort}: returns same input table, rows in different order
\textbf{Rename}: returns same table, one column with a different name
\textbf{Extend}: returns same table, additional column with computed value
\textbf{Groupby}: returns table, one row for each group of input records
%-----------------
\subsubsection{Two Table Operators}
\textbf{Product}: returns table containing all possible combinations of records from input tables
\textbf{Join}: connects tables together. Equivalent to selection of a product
\textbf{Semijoin}: returns table with records from input table1 that match some record in input table2
\textbf{Antijoin}: returns table with records from input table1 that do not match records in input table2
\textbf{Union}: returns table with records from both tables
\textbf{Outer Join}: returns table with records of join, and with non-matching records padded with nulls


%----------------------------------------
\section{SQL}


%--------------------------
\subsection{Normalization: 1, 2, 3 NF}
Organize fields and tables minimizing redundancy \& dependency. Usually split large tables into several smaller tables. And define relations between them.\\
\textbf{1. The key}. A table is in first normal form if it contains no repeating fields/columns. $phone$\\
\textbf{2. The whole key}. Second normal form refactors tables that have columns that do not depend on whole primary key. $course\ name$\\
\textbf{3. Nothing but the key}. Every non-key attribute must provide a fact about the key, the whole key, and nothing but the key. $birthday$

%--------------------------
\subsection{Create}
CREATE TABLE 'assignment2'.'page' (\\
\ \ \ \ 'id' INT NOT NULL AUTO\_INCREMENT,\\
\ \ \ \ 'title' VARCHAR(255) NULL,\\
\ \ \ \ 'created' DATE NULL,\\
\ \ \ \ 'website\_id' INT NULL,\\
\ \ \ \ 'user\_agreement' TINYINT NULL DEFAULT 1,\\
\ \ \ \ PRIMARY KEY ('id'),\\
\ \ \ \ \# INDEX 'page\_website\_composition\_idx' ('website\_id' ASC),\\
\ \ \ \ CONSTRAINT 'composition' FOREIGN KEY ('website\_id')\\
\ \ \ \ REFERENCES 'assignment2'.'website' ('id')\\
\ \ \ \ ON DELETE CASCADE ON UPDATE CASCADE\\
);\\
%--------------------------
\subsection{Trigger}
USE 'assignment2'\$\$\\
DROP TRIGGER IF EXISTS 'widget\_BEFORE\_INSERT';\\
DELIMITER \$\$\\
CREATE DEFINER = CURRENT\_USER TRIGGER (-) 'assignment2'.'widget\_BEFORE\_INSERT' BEFORE INSERT (-) ON 'widget' FOR EACH ROW\\
BEGIN\\
IF NEW.type = `heading' THEN\\
\ \ \ \ SET NEW.size = 2;\\
END IF;\\
END\$\$\\
DELIMITER;\\
%--------------------------
\subsection{Insert}
\textbf{INSERT INTO} 'assignment2'.'page' ('id', 'title', 'created', 'website\_id') \textbf{VALUES} (123, `Home', `2018-09-05', 567);
%--------------------------
\subsection{Retrieve}
\textbf{SELECT} DISTINCT t.column\_name(s) \textbf{FROM} table\_name t;\\
\textbf{SELECT}
	'Student \#' $||$ s.SId \textbf{AS} NewSId,
	s.SName,
	CAST(s.GradYear/10 as integer) $\times$ 10
		AS GradDecade\\
\textbf{FROM} STUDENT s;\\
\textbf{SELECT} a,
	\textbf{CASE} a
		\textbf{WHEN} 1 \textbf{THEN} 'one'
		\textbf{WHEN} 2 \textbf{THEN} 'two'
		\textbf{ELSE} 'other'
	\textbf{END}
\textbf{FROM} test;\\
Q69 = \textbf{SELECT} s.*, extract(YEAR, current\_date) $-$ s.GradYear \textbf{AS} AlumYrs
\textbf{FROM} STUDENT s;\\
\textbf{SELECT} q.*,
\textbf{if}(q.AlumYrs$>$0, 'alum', 'in school') \textbf{AS} GradStatus
\textbf{FROM} Q69 q;\\

%--------------------------
\subsection{JOIN}
\textbf{SELECT} column\_name(s) \textbf{FROM} table1 \textbf{JOIN} table2 \textbf{ON} table1.col1 = table2.col2;\\
\textbf{SELECT} s.SName, k.Prof \textbf{FROM} STUDENT s, ENROLL e, SECTION k \textbf{WHERE} s.SId = e.StudentId \textbf{AND} e.SectionId = k.SectId;\\
\textbf{LEFT JOIN} returns all rows from left table, even if there are no matches in the right table.\\
\textbf{RIGHT JOIN} returns all rows from right table, even if no matches in left table.\\
\textbf{FULL JOIN} returns rows when there is a match in either of the tables.\\


%--------------------------
\subsection{Update}
\textbf{UPDATE} page\_role pr1 JOIN page\_role pr2 \\
\textbf{SET} pr1.role = pr2.role, pr2.role = pr1.role\\
\textbf{WHERE} pr1.develope\_id = (SELECT id
        FROM
            person p\\
        WHERE
            p.username = 'charlie')
        AND pr2.developer\_id = (SELECT 
            id
        FROM
            person p
        WHERE
            p.username = 'bob')
        AND pr1.page\_id = (SELECT 
            page.id
        FROM
            page
                JOIN
            website w ON page.website\_id = w.id
        WHERE
            page.title LIKE '\%Home\%'
                AND w.name = 'CNET')
        AND pr2.page\_id = (SELECT 
            page.id
        FROM
            page
                JOIN
            website w ON page.website\_id = w.id
        WHERE
            page.title LIKE '\%Home\%'
                AND w.name = 'CNET');

%--------------------------
\subsection{Delete}
\textit{(Remove the last widget in the Contact page.)}\\
\textbf{DELETE FROM} widget \textbf{WHERE} widget.id IN\\ 
\ \ \ \ (SELECT * FROM\\
\ \ \ \ \ \ \ \ \ (SELECT wd.id FROM widget wd\\
\ \ \ \ \ \ \ \ \ JOIN page ON wd.page\_id = page.id\\
\ \ \ \ \ \ \ \ \ AND page.title = 'Contact') AS wdid)\\
\ \ \ \ AND widget.order = \\
\ \ \ \ \ \ \ \ \ (SELECT o FROM (SELECT 
            MAX(widget.order) AS o FROM
            widget JOIN page ON widget.page\_id = page.id
            AND page.title = 'Contact')\\
\ \ \ \ \ \ \ \ \ AS ord) LIMIT 1;


%--------------------------
\subsection{Group By, Having}
SELECT column\_name, aggregate\_function(column\_name)\\
FROM table\_name WHERE column\_name operator value\\
GROUP BY column\_name\\
HAVING aggregate\_function(column\_name) operator value

%--------------------------
\subsection{Order By, Limit}
SELECT column\_name(s)
FROM table\_name
\textbf{ORDER BY} column\_name(s) \textbf{ASC/DESC} \textbf{LIMIT} int\_number;



%--------------------------
\subsection{String}
\textbf{lower}/\textbf{upper} \ turn characters lower/upper case\\
\textbf{trim} \ remove leading and trailing spaces\\
\textbf{char\_length} \ return number of characters in string\\
\textbf{substring}('' from 2 for 4) \ Extract a specified substring\\
\textbf{current\_user} \ return the name of the current user\\
\textbf{$||$} \ concatenate two strings\\
\textbf{like} \ Match a string against a pattern. \% zero or more characters.
\_ exactly one character.
[charlist] Any single character in charlist.
[\^{}charlist] or [!charlist] Any single character not in charlist.

%--------------------------
\subsection{Date}
\textbf{NOW()} the current date and time.
\textbf{CURDATE()} the current date.
\textbf{CURTIME()} the current time.
\textbf{DATE()} Extracts the date part of a date or date/time expression.
\textbf{EXTRACT()} a single part of a date/time.
\textbf{DATE\_ADD()} Adds a specified time interval to a date.
\textbf{DATE\_SUB()} Subtracts a specified time interval from a date.
\textbf{DATEDIFF()} the number of days between two dates.
\textbf{DATE\_FORMAT()} Displays date/time data in different formats.


\subsection{Triggers, Functions \& Procedures}
\textbf{Triggers}; Change Data; \textbf{Never }Return Data; \textbf{Event Driven} When Called.\\

\textbf{Functions} \textbf{No} Change Data;  \textbf{Always} Return Data; \textbf{Part of Statement} When Called.\\

\textbf{Procedures} Change Data; \textbf{Sometimes} Return Data; \textbf{Executed} When Called.\\

%--------------------------
\subsection{Other}
SELECT column\_name(s)
FROM table\_name
WHERE column\_name
\textbf{BETWEEN} value1 \textbf{AND} value2;\\
SELECT column\_name(s)
FROM table\_name
WHERE column\_name
\textbf{NOT BETWEEN} value1 \textbf{AND} value2;\\
SELECT column\_name(s)
FROM table\_name
WHERE column\_name \textbf{IN} (value1, value2, ...);\\
$<>$ Not equal. $>=$ Greater than/equal. $<=$ Less than/equal.


%----------------------------------------
\section{JDBC}
%--------------------------
\subsection{Connect to JDBC}
\begin{lstlisting}[language=Java]
import java.sql.*;
public class StudentMajor {
    public static void main(String[] args) {
        Connection conn = null;
     try {
        Class.forName("com.mysql.jdbc.Driver");
        conn = DriverManager.getConnection(
        "jdbc:mysql://localhost:3306/studentdb", "root", "password");
\end{lstlisting}





%--------------------------
\subsection{Query}

\begin{lstlisting}[language=Java]
	...
	Statement  statement = conn.createStatement();
	String sql = "SELECT SName, DName FROM DEPT, STUDENT WHERE MajorId = DId";
	ResultSet result = statement.executeQuery(sql);
	...
\end{lstlisting}






%--------------------------
\subsection{ResultSet}

\begin{lstlisting}[language=Java]
	System.out.println("Name\tMajor");
	while (result.next()) {
		String sname = result.getString("SName");
		String dname = result.getString("DName");
		System.out.println(...);
	}
	result.close();
\end{lstlisting}







%--------------------------
\subsection{Update}

\begin{lstlisting}[language=Java]
...
	Statement statement = conn.createStatement();
	String sql ="UPDATE STUDENT SET MajorId=30 WHERE SName = `amy' ";
	statement.executeUpdate(sql);
	System.out.println("Amy is now a drama major.");
...
\end{lstlisting}



%--------------------------
\subsection{Disconnect}

\begin{lstlisting}[language=Java]
	try { ...
	} catch(SQLException e) {
		e.printStackTrace();
	} finally {
		try { 
		    if (conn != null) conn.close();
		} catch (SQLException e) {
			e.printStackTrace();
			} 
		} 
	}
\end{lstlisting}

\subsection{Other}
ResultSet represents a set of records resulting from a query.\\
A SELECT statement used in a FROM clause is sometimes referred to as inline view.\\
VIEW can be written in terms of select statement.\\
All operations performed on a view actually affect the base table of the view.\\
View can be written in terms of other views.\\
User-defined functions cannot be used to perform actions that modify the database state.\\
java.sql and javax.sql contain the JDBC classes.\\
PreparedStatement can execute parameterized queries and take care of SQL injnjection.\\
JDBC stands for Java DataBase Connectivity.\\
JDBC is an API which defines how a client may access a relational databases.\\
JDBC allows multiple implementations to exist and be used by the same application.\\
Implement generalization using separate tables for each class, e.g., the \textbf{normalized }strategy. Name the constraint on the foreign keys using the following pattern: subclass\_superclass\_generalization, where subclass and superclass are the subclass and superclass in the diagram.\\
Implement generalization using a single table, e.g., the \textbf{denormalized} strategy. Use a new field called DTYPE to discriminate for the type.\\
\subsection{}
\begin{lstlisting}

<?xml version=”1.0” encoding=”UTF-8”?>
<xsl: stylesheet version=”1.0”
Xmlns:xsl=http://www.w3.org/1999/XSL/Transform>
<xsl:output method=”xml”> indent=”yes”/>
<xsl: template match=”flight”>
<reservations>
<xsl:>
</reservations>
</xsl:template>
<xsl:template match=”flights”/flight/reservations/reservation”>
<reservation>
<xsl:attribute name=”transactionType”>
<xsl:value-of select=”transactionType”/></xsl:attribute>
<reservationId><xsl:value-of select=”@reservationId”/>
</reservationId>
<bookingAgent><xsl: value-of select=”@bookingAgent”/>
<bookingAgent>
</reservation>
</xsl:template>
</xsl:stylesheet>

\end{lstlisting}

\begin{lstlisting}
public Movie findMovieByImbd(String id) {
    Movie movie = null;
    java.sql.Connection connection = null;
    PreparedStatement statement = null;
    ResultSet results = null;
    try {
        ...
        connection = ...;
        String sql = "SELECT * FROM movies WHERE imbdId = ?";
        statemetn = conenction.prepareStatement(sql);
        statement.setString(1, id);
        results = statement.executeQuery();
        if(results.next()){
            String title = results.getString("title");
            String imbdId = results.getString("imbdId");
            String plot = results.getString("plot");
            String poster = results.getString("poster");
            movie = new Moive(imbdId, title, plot, poster);
            }
        } catch (ClassNotFOundException e){
            ...}
        return movie;
    }
\end{lstlisting}




\end{multicols}
\end{document}
