\documentclass[10pt,landscape]{article}
\usepackage{multicol}
\usepackage{calc}
\usepackage{ifthen}
\usepackage[landscape]{geometry}
\usepackage{hyperref}
\usepackage{amsmath}
\usepackage{graphicx}
\usepackage{subfigure}
\usepackage{float}
\usepackage{ulem}
\usepackage{bm}
\usepackage{anysize}
\usepackage{MnSymbol}
\usepackage{listings}
\usepackage{color}
\usepackage{xcolor}
\definecolor{dkgreen}{rgb}{0,0.6,0}
\definecolor{gray}{rgb}{0.5,0.5,0.5}
\definecolor{mauve}{rgb}{0.58,0,0.82}
\lstset{frame=tb,
     language=Java,
     aboveskip=3mm,
     belowskip=3mm,
     showstringspaces=false,
     columns=flexible,
     basicstyle = \ttfamily\small,
     numbers=none,
     numberstyle=\tiny\color{gray},
     keywordstyle=\color{blue},
     commentstyle=\color{dkgreen},
     stringstyle=\color{mauve},
     breaklines=true,
     breakatwhitespace=true,
     tabsize=3
}

% This sets page margins to .5 inch if using letter paper, and to 1cm
% if using A4 paper. (This probably isn't strictly necessary.)
% If using another size paper, use default 1cm margins.
\ifthenelse{\lengthtest { \paperwidth = 11in}}
	{ \geometry{top=.5in,left=.5in,right=.5in,bottom=.5in} }
	{\ifthenelse{ \lengthtest{ \paperwidth = 297mm}}
		{\geometry{top=1cm,left=1cm,right=1cm,bottom=1cm} }
		{\geometry{top=1cm,left=1cm,right=1cm,bottom=1cm} }
	}

% Turn off header and footer
\pagestyle{empty}
 

% Redefine section commands to use less space
\makeatletter
\renewcommand{\section}{\@startsection{section}{1}{0mm}%
                                {-1ex plus -.5ex minus -.2ex}%
                                {0.5ex plus .2ex}%x
                                {\normalfont\large\bfseries}}
\renewcommand{\subsection}{\@startsection{subsection}{2}{0mm}%
                                {-1explus -.5ex minus -.2ex}%
                                {0.5ex plus .2ex}%
                                {\normalfont\normalsize\bfseries}}
\renewcommand{\subsubsection}{\@startsection{subsubsection}{3}{0mm}%
                                {-1ex plus -.5ex minus -.2ex}%
                                {1ex plus .2ex}%
                                {\normalfont\small\bfseries}}
\makeatother

% Define BibTeX command
\def\BibTeX{{\rm B\kern-.05em{\sc i\kern-.025em b}\kern-.08em
    T\kern-.1667em\lower.7ex\hbox{E}\kern-.125emX}}

% Don't print section numbers
\setcounter{secnumdepth}{0}


\setlength{\parindent}{0pt}
\setlength{\parskip}{0pt plus 0.5ex}


% -----------------------------------------------------------------------

\begin{document}

\raggedright
\footnotesize
\begin{multicols}{3}


% multicol parameters
% These lengths are set only within the two main columns
%\setlength{\columnseprule}{0.25pt}
\setlength{\premulticols}{1pt}
\setlength{\postmulticols}{1pt}
\setlength{\multicolsep}{1pt}
\setlength{\columnsep}{2pt}

% $_{by\ \text{Jiyu Tian}}$
 
\begin{center}
     \Large{\textbf{Database Final}} \\
\end{center}


%----------------------------------------
\section{Transactions}
\begin{itemize}
    \item Atomic - all or nothing
    \item Consistency - transactions ensure database transitions between consistent states
    \item Isolation - concurrent execution must be unaware of each other
    \item Durability - committed transactions remain so, despite power loss or crashes
\end{itemize}



Several clients access DB server concurrently. Must not interfere with each other.\\
\textbf{Dirty reads} Reading uncommitted Data.\\
\textbf{Non-repeatable Reads }Unexpected changes to existing record.\\
\textbf{Phantom Reads} Unexpected Changes to Number of Records.\\




%----------------------------------------
%----------------------------------------
\subsection{Isolation levels}
\textbf{Read-Uncommitted}
Lowest isolation level. 
Dirty reads are allowed. 
One transaction may see changes not-yet-committed by other transactions.
Fastest.\\
\textbf{Read Committed}
Guarantees data read is committed.
Restricts reader from reading intermediate, uncommitted, 'dirty' read.
Data is free to change after read.
If transaction re-issues read, it might have changed.
Read/Write locks acquired on selected data.
Released at end of transaction.
Read locks released as soon as SELECT is performed.
No range-locks are managed.
Non-repeatable/Phantom reads can occur
Slower.\\
\textbf{Repeatable Reads}
Read/Write locks acquired on selected data.
Released at end of the transaction.
No range-locks managed.
Phantom reads can occur.
Even slower.\\
\textbf{Serializable}
Highest isolation level.
Read/Write locks acquired on selected data.
Released at end of transaction.
Range-locks acquired when SELECT uses ranged WHERE clause to avoid phantom reads.
Locks everything, no problems, slowest.

\begin{equation*}
\begin{array}{cccc}
& DR & NRR & PR \\
RU & \text{Unsafe} & \text{Unsafe} & \text{Unsafe}\\
RC & \text{Safe} & \text{Unsafe} & \text{Unsafe}\\
RR & \text{Safe} & \text{Safe} & \text{Unsafe}\\
S & \text{Safe} & \text{Safe} & \text{Safe}
\end{array}
\end{equation*}

%----------------------------------------
\section{Indexing}
Query performance depends on table organization. Unsorted list $O(n)$; sorted \& organized $O(log_2n)$; hash data $O(1)$. 
More organization, more efficiency.\\
An \textbf{index} is a smaller, easier to maintain, copy of original table; file with index values, record identifiers (RIDs) referencing original records. Unique keyword identify index as referring to a key.\\


%----------------------------------------
%----------------------------------------
\subsection{Unique and Non-Unique Indexes}
Queries on \textbf{unique indexes} refer to single record, i.e., searching for SID\_IDX for SId=10, refers to single record.\\

Queries on \textbf{non unique indexes} may refer to several records, i.e., searching for MAJOR\_IDX for MajorId=10, may refer to several records


%----------------------------------------
%----------------------------------------
\subsection{Indexes Not Always Best}
Sometimes it's faster to just use linear search.
Non-unique indexes are less efficient than unique indexes, i.e., on primary keys.
The more matching records, the less efficient indexes are.


%----------------------------------------
%----------------------------------------
\subsection{Composite Indixes}
If most select queries use both indexes, then consider a composite index.
Column order defines sort-priority order.
Add ASC or DESC next to column to define sort order.
Easier to maintain one index on two cols then two separate indexes.
Order of columns matters: Two indexes $\to$ two choices
If both indexes in query, order doesn't matter.
If only one index in where, then order matters.









%----------------------------------------
\section{REST}
Representational state transfer (REST) is a set of architectural conventions for creating Web services. 
HTTP Web services conventions on HTTP operations (state transition)
\begin{itemize}
    \item POST: Create new instances
    \item GET: Read existing instances
    \item PUT: Update existing instances
    \item DELETE: Delete existing instances
\end{itemize}
RESTful Web services map HTTP URL request path patterns to the structure of data served by a Web service.


%----------------------------------------
%----------------------------------------
\subsection{Mapping URLs to a data model}
$A \blacklozenge- B \blacklozenge- C \blacklozenge- D $\\
Class A represents all instances of type A, referred to as entities. Instances can be implemented as records in a database or object instances in a running program. \\
\textbf{POST}		/A			Create new entity of type A. Data in HTTP body\\
\textbf{GET}		/A			Retrieve all instances of entity type A, e.g., array\\
\textbf{GET}		/A/123		Retrieve single instance, whose ID is 123\\
\textbf{PUT}		/A/234	Update instance, whose ID is 234. Data in body\\
\textbf{DELETE}		/A/345		Delete existing instance, whose ID is 345\\
URL paths hierarchy allows navigating through complex data structures.\\

\textbf{POST}	/A/123/B	Create new instance of B related to instance of A whose ID is 123. Data for new B in HTTP body\\
\textbf{GET}		/A/234/B	Retrieve all instances of B related to instance of A whose ID is 234. As an array or list.\\

Referencing nested data instances has two alternatives.\\
\textbf{GET} /A/123/B/234 Retrieve instance B with ID 234 related to instance A with ID 123. Useful if service needs context of A\\
\textbf{GET} /B/234	Retrieve instance B with ID 234, regardless of A. Useful when context of A is unnecessary.


%----------------------------------------
%----------------------------------------
\subsection{JPA}
JPA maps Java classes to database tables, properties to fields, and instances to records.\\

\textbf{@Entity} annotate Java Class. For JPA to correctly map database tables into Java classes the concept of Entity was created. An Entity must be created to support the same database table structure; thus JPA handles table data modification through an entity.\\
\textbf{@Id} annotation defines an identifier indicating the member field below is the primary key of current entity.\\
\textbf{@GeneratedValue} annotation indicates that the identifier value should be automatically generated, and the specified strategy of IDENTITY indicates that an identity column should be used to generate the identifier. GenerationType.TABLE/ SEQUENCE/IDENTITY/AUTO.\\

%----------------------------------------
%----------------------------------------
\subsection{@RequestParam v.s. @ParthVariable}
\textbf{@PathVariable} is to obtain some placeholder from the URI. It identifies the pattern that is used in the URI for the incoming request. It extracts values from URI.\\
\textbf{@RequestParam} is to obtain an parameter from the URI, also known as query parameters. \\

%----------------------------------------
\section{MongoDB}
\begin{tabular}{@{}ll@{}}
Start database server  & $mongod$ \\
Connect to database &  $mongo$ \\
Create a database  & $use\ <database>$ \\
Operation & $db.<collection>.<cmd>(<data>)$ \\
List existing collections & $show\ collections$\\
List existing databases & $show\ dbs$\\
\end{tabular}

Commands includes \textbf{insert}(), \textbf{find}(), \textbf{update}(), \textbf{remove}().\\
\ \\
\textbf{insert}() inserts data into a collection. Inserting data creates collection if none existent. Data is formatted as JSON objects. \\
\ \\
\textbf{find}() matches all fields in first parameter.
Second parameter selects fields. Mongo adds unique identifier field $\_id$. 
Use the $pretty()$ command to pretty print queries.  $db.user.find().pretty()$. 
Use \textbf{ObjecId} to filter by primary key $\_id$.
Use sort() to sort query results.
ascending: 1; descending: -1.


%----------------------------------------
%----------------------------------------
\subsection{Examples}
\begin{lstlisting}
> db.section.find({seats: {$gt: 30}})
> db.section.find({
    $and: [
        {seats: {$gt: 30}},
        {seats: {$lt: 40}}
    ]})
> db.section.find().sort({seats: 1})
> db.section.find().sort({seats: -1})
> db.section.update({name: '02'}, 
    {$set: {seats: 22}})
> db.section.remove({name: '02'})
> db.student.update({username: 'bob'},
    {$push:{sections: 'SEC01'}})
\end{lstlisting}


%----------------------------------------
%----------------------------------------
\subsection{Modeling Generalization}
Model generalizations by flattening hierarchy and adding a type property, $aka$ denormalized strategy. Add a $Type$ field.


%----------------------------------------
%----------------------------------------
\subsection{Many/One to Many}
\begin{lstlisting}
> db.enrollment.insert({
    section: '5aa5bd55ab319d0f94089fd3',
    student: '5aa9515ad982cb8e72af70ce'
    })
> db.section.update({
    _id: ObjectId('5aa5bd55ab319d0f94089fd3')}, 
    {$set: {
        faculty: '5aa5ab73ab319d0f94089fc7'}
    })

\end{lstlisting}


%----------------------------------------
%----------------------------------------
\subsection{Embedded Objects}
\begin{lstlisting}
> db.modules.insert({name: 'module1',
    lessons: [{name: 'lesson1'}]});
> db.modules.update({name: 'module1'},
    {$push: {lessons: {name: 'lesson2'}}});
> db.modules.update({name: 'module1', 
    'lessons.name': 'lesson1'}, 
    {$set: {'lessons.$.topics': 
        [{name: 'topic1'}]}})
\end{lstlisting}
The positional \$ operator identifies an element in an array to update without explicitly specifying the position of the element in the array. 


%----------------------------------------
\section{Mongoose}
SchemaTypes: String, Number, Date, Buffer, Boolean, Mixed, ObjectId, Array, Decimal128, Map.

%----------------------------------------
%----------------------------------------
\subsection{db.js}
\begin{lstlisting}
module.exports = function () {
   const mongoose = require('mongoose');
   const databaseName = 'whiteboard';
   var   connectionString = 
'mongodb://localhost/';
   connectionString += databaseName;
   mongoose.connect(connectionString);
};
\end{lstlisting}

%----------------------------------------
%----------------------------------------
\subsection{user.schema.server.js}
\begin{lstlisting}
const mongoose = require('mongoose');
const userSchema = mongoose.Schema({
   firstName: String,
   lastName: String,
   role: {
       type: String,
       enum:["Student", "Faculty"], 
       default: "Student"
       }
}, {collection: 'user'});
module.exports = userSchema;
\end{lstlisting}


%----------------------------------------
%----------------------------------------
\subsection{user.model.server.js}
\begin{lstlisting}
const mongoose = require('mongoose');
const userSchema = require('./user.schema.server');
const userModel = mongoose.model('UserModel', userSchema);

findAllUsers = () =>
    userModel.find();
findUserById = userId =>
    userModel.findById(userId)
findUserByUsername = username =>
    userModel.find({username: username})
  
module.exports = {findAllUsers, findUserById, findUserById};
\end{lstlisting}



%----------------------------------------
%----------------------------------------
\subsection{enrollment.schema.server.js}
\begin{lstlisting}
const mongoose = require('mongoose');
const enrollmentSchema = mongoose.Schema({
 student: {ref: 'UserModel',
  type: mongoose.Schema.Types.ObjectId},
 section: {ref: 'SectionModel',
  type: mongoose.Schema.Types.ObjectId}
});
module.exports = enrollmentSchema;
\end{lstlisting}

%----------------------------------------
%----------------------------------------
\subsection{test.js}
\begin{lstlisting}
require('./db')();
var userDao = require('./models/user.dao.server');
userDao.find((err, users) => {
   console.log(users);
});
\end{lstlisting}




%----------------------------------------
\section{XML}
XML is a markup language that defines a set of rules for encoding documents in a format that is both human-readable and machine-readable. \\
\begin{itemize}
    \item XML: Extendable Markup Language
    \item Language: vocabulary, syntax, grammar
    \item Markup: annotations providing meta-data, meaning, semantics, context
    \item Extendable: can be used in any context
    \item Elements: annotate data
    \item Example: Persistence.xml, pom.xml - persitence, persistence-unit, provide, class
    \item Element names are user friendly but easy for computer parsing
\end{itemize}

XML document constraints can be expressed as XML Schema documents. 
XML schema describe valid XML document structures, data types, and relations.
Document Type Definitions (DTDs) is a popular XML schema language.
DTDs can be used to validate XML documents.


%----------------------------------------
%----------------------------------------
\subsection{Example}

\begin{lstlisting}
<Enrollments gradyear="2009">
	<CourseTaken sname="ian" title="pop culture"
	yeartaken="2007" grade="B+" />
	<CourseTaken sname="ben" title="shakespeare"
	yeartaken="2006" grade="A-" />
</Enrollments>
\end{lstlisting}

\begin{itemize}
    \item Enrollments: root element, grouping element
    \item CourseTaken – one element per course
    \item Tags
    \item Bracketed tags and Empty tags
    \item Attributes
    \item Semantics can be captured in tags and / or attributes
    \item Sibling meanings can be captured in parent's context
\end{itemize}
All tags must be closed.
Tags can be mixed case, but matching opening and closing tags must be capitalized the same way.
Tags (or elements) can be arbitrarily nested, describing complex data structures.
Tags, as well as nested tags, must be closed in the correct order.
\begin{lstlisting}
<GraduatingStudents>
	<Student>
		<SName>ian</SName>
		<GradYear>2009</GradYear>
		<Courses>
			<Course>
				<Title>calculus</Title>
				<YearTaken>2006</YearTaken>
				<Grade>A</Grade>
			</Course>
			<Course>
				<Title>shakespeare</Title>
				<YearTaken>2006</YearTaken>
				<Grade>C</Grade>
			</Course>…
		</Courses>
	</Student>...
</GraduatingStudents>
\end{lstlisting}




%----------------------------------------
%----------------------------------------
\subsection{JAXB}

\textbf{Create Data Model}

\begin{lstlisting}
Student.java
public class Student {
		private String id;
	private String firstName;
		private String lastName;
}
student.xml
<student id="123">
    <firstName>Alice</firstName>
    <lastName>Wonderland</lastName>
</student>
\end{lstlisting}

\textbf{Add Constructors}
\begin{lstlisting}
public Student(String id, String firstName,...) {
	super();
	this.id = id;
	this.firstName = firstName;
	this.lastName = lastName;
}
public Student() { // required
	super();
}
\end{lstlisting}

\textbf{Add Setters and Getters}
\begin{lstlisting}
public String getId() {
return id;
}
public void setId(String id) {
this.id = id;
}
public String getFirstName() {
return firstName;
}
public void setFirstName(String firstName) {
this.firstName = firstName;}...etc...
\end{lstlisting}


\textbf{Annotate Setters with JAXB}

\begin{lstlisting}
import javax.xml.bind.annotation.*;

@XmlRootElement
public class Student {
	@XmlAttribute
	public void setId(String id) {
		this.id = id;
	}
	@XmlElement
	public void setFirstName(String firstName) {
		this.firstName = firstName;
	} ...etc...
}
\end{lstlisting}

\textbf{Create JAXB Context \& Marshaller}

\begin{lstlisting}
public static void main(String[] args) {
 Student alice = new Student("123","alice"...);
 try {
  JAXBContext jaxbContext = 
JAXBContext.newInstance(Student.class);
  Marshaller jaxbMarshaller = 
jaxbContext.createMarshaller();
 } catch (JAXBException e) {
  e.printStackTrace();
 }
}
\end{lstlisting}


\textbf{Marshal Object to File}

\begin{lstlisting}
 try {
  Marshaller jaxbMarshaller = ...
  jaxbMarshaller
    .setProperty(Marshaller
      .JAXB_FORMATTED_OUTPUT, true);
  File xmlFile = new File("src/.../alice.xml");
  jaxbMarshaller.marshal(alice, xmlFile);
} catch (JAXBException e) {
  e.printStackTrace();
}
\end{lstlisting}



%----------------------------------------
%----------------------------------------
\subsection{XSLT}
When sharing data with non users, often need to generate / transform to suitable representation.
Use XSLT to generate / transform XML documents.
Stands for eXtendable Stylesheet Language Transformation
Declarative programming.
XSLT program - stylesheet, written in XML.
Made up of elements / commands to transform.





%----------------------------------------
%----------------------------------------
%----------------------------------------
\subsubsection{XML to HTML}
XML just contains data, like a database
It does not contain info on how data should be used
We can transform XML into many other formats: HTML, SQL, PDF, DOC, ETC, ....

\begin{lstlisting}
<xsl:stylesheet xmlns:xsl="" version="2.0">
<xsl:output method="html"/>

<xsl:template match="GraduatingStudents">
 <html><body>
<table border="1" cellpadding="2">
	<tr>	<th>Name</th> <th>Grad Year</th>
			<th>Courses</th> </tr>
		<xsl:apply-templates select="Student"/>
 	</table>
 </body></html>
</xsl:template>
\end{lstlisting}




\begin{lstlisting}
<xsl:template match="Student">
<tr> <td><xsl:value-of select="SName"/>
</td>
		<td><xsl:value-of select="GradYear"/>
</td>
<td><xsl:apply-templates
select="Courses"/></td>
</tr>
</xsl:template>
\end{lstlisting}



\begin{lstlisting}
<xsl:template match="Courses">
	<table border="0" cellpadding="2"
width="100%">
<tr>	<th width="50%">Title</th>
		<th>Year</th> <th>Grade</th> </tr>
<xsl:apply-templates select="Course"/>
	</table>
</xsl:template>
\end{lstlisting}




\begin{lstlisting}
<xsl:template match="Course">
	<tr>	<td><xsl:value-of select="Title"/>
</td>
		 		<td><xsl:value-of select="YearTaken"/>
</td>
<td><xsl:value-of select="Grade"/>
</td>
  </tr>
</xsl:template>
</xsl:stylesheet>
\end{lstlisting}




%----------------------------------------
%----------------------------------------
%----------------------------------------
\subsubsection{Transform from normalized to denormalized}
Consider this hierarchical (normalized) XML file:
\begin{lstlisting}
<AllProjects>
<project project_id=”2″>
<name>Brand New Browser</name>
<phase id=”2.A”>
<desc>Firefox Cleanup</desc>
<task priority=”1″><resource>M</resource></task>
<task priority=”2″><resource>M</resource></task>
</phase>
<phase id=”2.B”>
<desc>Rebranding</desc>
<task priority=”1″><resource>B007</resource></task>
<task priority=”2″><resource>SC</resource></task>
</phase>
</project>
………..
</AllProjects>
\end{lstlisting}


Consider this corresponding denormalized XML file:
\begin{lstlisting}

<AllTasks>
<task>
<projectid>2</projectid>
<projectname>Brand New Browser</projectname>
<phaseid>2.A</phaseid>
<phasedesc>Firefox Cleanup</phasedesc>
<priority>1</priority>
<resource>M</resource>
</task>
<task>
<projectid>2</projectid>
<projectname>Brand New Browser</projectname>
<phaseid>2.A</phaseid>
<phasedesc>Firefox Cleanup</phasedesc>
<priority>2</priority>
<resource>M</resource>
</task>
…
<AllTasks>

\end{lstlisting}


This denormalization can be done very easily using an XSLT. The following XSLT conveys some concepts used in this:


\begin{lstlisting}
<?xml version = “1.0” encoding = “UTF-8”?>
<xsl:transform xmlns:xsl = “http://www.w3.org/1999/XSL/Transform” version = “1.0”>
<xsl:template match = “AllProjects”>
<AllTasks>
<xsl:apply-templates select = “project/phase/task”/>
</AllTasks>
</xsl:template>
<xsl:template match = “task”>
<xsl:element name = “task”>
<projectid><xsl:value-of select=”../../@project_id”/></projectid>
<projectname><xsl:value-of select=”../../name”/></projectname>
<phaseid><xsl:value-of select=”../@id”/></phaseid>
<phasedesc><xsl:value-of select=”../desc”/></phasedesc>
<priority><xsl:value-of select=”@priority”/></priority>
<resource><xsl:value-of select=”resource”/></resource>
</xsl:element>
</xsl:template>
</xsl:transform>
\end{lstlisting}







%----------------------------------------
%----------------------------------------
%----------------------------------------
\subsubsection{XML 2 HTML}

\textbf{students.xml}
\begin{lstlisting}
<?xml version="1.0" encoding="UTF-8"?>
<students>
<student id="123">
<firstName>Alice</firstName>
     <lastName>Wonderland</lastName>
     <username>alice</username>
</student>
<student id="234">
     <firstName>Bob</firstName>
     <lastName>Marley</lastName>
     <username>bob</username>
</student>...		</students>
\end{lstlisting}


\textbf{studentsXml2Html.xsl}
\begin{lstlisting}
<?xml version="1.0" encoding="UTF-8"?>
<xsl:stylesheet version="1.0" xmlns:xsl= "http://www.w3.org/1999/XSL/Transform">
	<xsl:output method="html"/>
	<xsl:template match="/">
		<H1>Students</H1>
		<TABLE><THEAD><TR><TH>Id</TH><TH>Username</TH>
    <TH>First Name</TH><TH>Last Name</TH>
    </TR></THEAD>
		<TBODY><xsl:apply-templates/></TBODY></TABLE>
	</xsl:template>
\end{lstlisting}



\textbf{studentsXml2Html.xsl}

\begin{lstlisting}
	<xsl:template match="students/student">
		<TR>
			<TD><xsl:value-of select="@id"/></TD>
			<TD><xsl:value-of select="username"/></TD>
			<TD><xsl:value-of select="firstName"/></TD>
			<TD><xsl:value-of select="lastName"/></TD>
		</TR>
	</xsl:template>
</xsl:stylesheet>
\end{lstlisting}







%----------------------------------------
%----------------------------------------
%----------------------------------------
\subsubsection{Selecting Nodes in XPath}

\begin{tabular}{@{}ll@{}}
 \textbf{Path Expression} & \textbf{Result}\\
nodename & Selects all nodes w/ name "nodename"\\
/ & Selects from the root node\\
// & Selects nodes in the document from the \\
 & current node that match the selection \\
 & no matter where they are\\
. & Selects the current node\\
.. & Selects the parent of the current node\\ 
@ & Selects attributes\\
\end{tabular}

\begin{tabular}{@{}ll@{}}
 \textbf{Path Expression} & \textbf{Result}\\
bookstore & Selects all nodes w/ name "bookstore"\\
/bookstore & Selects the root element bookstore\\
bookstore/book & Selects all book elements that are children \\
 & of bookstore\\
//book & Selects all book elements no matter where \\
 & they are in the document\\
bookstore//book & Selects all book elements that are \\
 & descendant of the bookstore element\\
//@lang & Selects all attributes named lang\\

\end{tabular}

\begin{tabular}{@{}ll@{}}
 \textbf{Path Expression} & \textbf{Result}\\
/bookstore/book[1]  & Selects the first book element \\
  & that is the child of the\\
  & bookstore element\\
/bookstore/book[last()] & Selects the last book \\
/bookstore/book[last()-1] & Selects the last but one\\
/bookstore/book[position()$<$3] & Selects the first two book\\
//title[@lang]  & Selects all the title elements \\
  & that have an attribute \\
    & named lang\\
//title[@lang='en'] & Selects all title elements that\\ 
  & have an attribute lang with\\
   & a value of 'en'\\
/bookstore/book[price$>$35] & Selects all the book elements\\
  & that have a price than 35\\
/bookstore/book[price$>$35]/title & Selects all title elements with\\
  & a value greater than 35\\

\end{tabular}




%----------------------------------------
%----------------------------------------
\subsection{DOM}
The Document Object Model (\textbf{DOM}) is a programming interface for HTML and XML documents. It represents the page so that programs can change the document structure, style, and content. The DOM represents the document as nodes and objects. That way, programming languages can connect to the page.



\end{multicols}
\end{document}



%----------------------------------------
\section{Other}
\begin{lstlisting}
<?xml version=”1.0” encoding=”UTF-8”?>
<xsl: stylesheet version=”1.0”
Xmlns:xsl=http://www.w3.org/1999/XSL/Transform>
<xsl:output method=”xml”> indent=”yes”/>
<xsl: template match=”flight”>
<reservations>
<xsl:>
</reservations>
</xsl:template>
<xsl:template match=”flights”/flight/reservations/reservation”>
<reservation>
<xsl:attribute name=”transactionType”>
<xsl:value-of select=”transactionType”/></xsl:attribute>
<reservationId><xsl:value-of select=”@reservationId”/>
</reservationId>
<bookingAgent><xsl: value-of select=”@bookingAgent”/>
<bookingAgent>
</reservation>
</xsl:template>
</xsl:stylesheet>
\end{lstlisting}





%----------------------------------------
%----------------------------------------
\subsection{}